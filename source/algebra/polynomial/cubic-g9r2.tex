% $date: 2018-11-06 p3
% $date$groups:
%   g9r2: 2018-11-06 p3

\worksheet{Многочлен третьей степени: график и корни}

% $authors:
% - Александр Савельевич Штерн

%Два типа графика приведённого многочлена третьей степени; центральная
%симметричность графика; выделение полного куба; целые и рациональные корни;
%число корней.

\begin{problems}

\item
Докажите, что любой многочлен третьей степени принимает как положительные, так
и~отрицательные значения.

\item
Найдите хотя бы один корень уравнения
$x^3 - 3 \sqrt{6} x + 2 \sqrt{2} + 3 \sqrt{3} = 0$.
Сколько корней имеет это уравнение?

\item
Две параллельные прямые проведены так, что каждая из них пересекает график
кубического многочлена в трёх точках.
Первая в точках $A$, $D$, $E$.
Вторая~--- в точках $B$, $C$, $F$.
Докажите, что длина проекции дуги $CD$ на ось абсцисс равна сумме длин проекций
дуг $AB$ и $EF$.

\item
На доске написано уравнение $x^3 + * x^2 + * x + * = 0$.
Два игрока по очереди заменяют звёздочки целыми числами, отличными от нуля.
Первый игрок выигрывает, если полученное уравнение имеет не менее двух
различных целых корней, в противном случае выигрывает второй игрок.
Кто выиграет при правильной игре?

\item
На координатной плоскости имеется квадрат со сторонами, параллельными осям
координат.
Этот квадрат поделён на $64$ равных квадратика прямыми, параллельными осям
координат.
Внутри квадрата движется точка, координаты которой в каждый момент времени $t$
вычисляются по формулам
$x = a t^3 + b t^2 + c t + d$,
$y = A t^3 + B t^2 + C t + D$.
Докажите, что среди этих $64$ квадратиков найдётся такой, внутри которого точка
не находилась ни в какой момент времени.

\item
У Феди есть три палочки.
Если из них нельзя сложить треугольник, Федя укорачивает самую длинную
из палочек на сумму длин двух других.
Если длина палочки не обратилась в нуль и треугольник снова нельзя сложить, то
Федя повторяет операцию, и~т.\,д.
Может ли этот процесс продолжаться бесконечно?

\item
Сумма трёх положительных чисел равна $10$, а сумма их квадратов больше $20$.
Докажите, что сумма их кубов больше $40$.

\item
Приведённый многочлен третьей степени имеет три вещественных положительных
корня.
Выпишите необходимое и достаточное условие на его коэффициенты, гарантирующее,
что эти корня являются длинами сторон некоторого треугольника.

\end{problems}


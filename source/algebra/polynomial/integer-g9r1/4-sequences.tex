% $date: 2018-11-07 p1
% $date$groups:
%   g9r1: 2018-11-07 p1

\worksheet{Многочлены, последовательности и рекуррентные соотношения}

% $authors:
% - Александр Савельевич Штерн

\begin{problems}

\item
Пусть $T_{k}(n)$ есть сумма всех произведений по $k$ чисел от $1$ до $n$.
\\
\subproblem
Докажите, что $T_{2}(n)$ есть многочлен.
\\
\subproblem
Докажите, что $T_{k}(n)$ является многочленом от $n$ при произвольном $k$.

\item
Докажите, что при любых натуральных $n$, $m$ выражение
\[
    \frac
        {(x^{n+1} - 1) (x^{n+2} - 1) \ldots (x^{n+m} - 1)}
        {(x - 1) (x^{2} - 1) \ldots (x^{m} - 1)}
\]
является многочленом (так называемые \emph{многочлены Гаусса}).

\item
Дана последовательность $a_{n}$.
Разрешается получать новые последовательности по следующим правилам:
отбрасывать несколько первых членов;
почленно складывать, вычитать, умножать и делить (если ни один из членов
стоящей в знаменателе последовательность никогда не обращается в ноль) любые
две имеющиеся последовательности.
Можно ли, действуя таким образом много раз, получить натуральный ряд
(полностью), если изначально имеется ряд квадратов?

\item
Тот же вопрос, но изначально имеется последовательность $a_{n} = n + \sqrt{2}$.

\item
Последовательность $a_{0}, a_{1}, a_{2}, \ldots$ задана условиями
$a_{0} = 0$,  $a_{n+1} = P(a_{n})$ ($n \geq 0$),
где $P(x)$~--- многочлен с целыми коэффициентами, $P(x) > 0$ при $x \geq 0$.
Докажите, что наибольший общий делитель любых двух членов этой
последовательности принадлежит этой последовательности.

\item
Дан многочлен $P(x)$ с целыми коэффициентами, причём для каждого
натурального $x$ выполняется неравенство $P(x) > x$.
Определим последовательность $\{ b_{n} \}$ следующим образом:
$b_{1} = 1$, $b_{k+1} = P(b_{k})$ для $k \geq 1$.
Известно, что для любого натурального $d$ найдется член
последовательности $\{ b_{n} \}$, делящийся на $d$.
Найдите все такие многочлены $P$.

\renewcommand\binom[2]{\mathrm{C}_{#1}^{#2}}

\item
Пусть $p$ и $q$~--- различные числа и $p + q = 1$.
Упростите следующее выражение так, чтобы исчез знак суммирования:
\[
    \binom{n}{0} - \binom{n}{1} p q + \binom{n}{2} p^{2} q^{2}
    - \ldots + (-1)^{i} \binom{n - i}{i} p^{i} q^{i} + \ldots
\]
(слагаемые суммируются до тех пор, пока биномиальный коэффициент корректно
определён).

\end{problems}


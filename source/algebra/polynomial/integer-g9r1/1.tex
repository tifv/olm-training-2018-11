% $date: 2018-11-05 p2
% $date$groups:
%   g9r1: 2018-11-05 p2

\worksheet{Многочлены с целыми коэффициентами}

% $authors:
% - Александр Савельевич Штерн

\begin{problems}

\item
Докажите, что многочлен $f(x) = x^{p-1} + x^{p-2} + x^{p-3} + \ldots + 1$ при
простом $p$ не раскладывается в~произведение многочленов меньшей степени
с~целыми коэффициентами.

\item
Докажите, что многочлен $(x - a_{1}) (x - a_{2}) \ldots (x - a_{n}) - 1$
не раскладывается в~произведение многочленов меньшей степени с~целыми
коэффициентами, если $a_{1}$, $a_{2}$, \ldots, $a_{n}$~--- различные целые
числа.

\item
Пусть $P(x) = a_{n} x^{n} + \ldots + a_{1} x + a_{0}$~--- многочлен с~целыми
коэффициентами, причём существует такое простое $p$, что\enspace
$p \mid a_{0}, a_{1}, \ldots, a_{k}$,\enspace
$p \nmid a_{k+1}$,\enspace
$p^2 \nmid a_{0}$.
Тогда любое разложение многочлена $P(x)$ на множители содержит многочлен
степени больше $k$.

\item
Трёхчлен с~целыми коэффициентами $a x^{2} + b x + c$ при всех целых $x$
является точной четвёртой степенью.
Докажите, что тогда $a = b = 0$.

\item
Трёхчлен $a x^{2} + b x + c$ при всех целых $x$ является точным квадратом.
Докажите, что этот многочлен является степенью линейного двучлена с~целыми
коэффициентами.

\item
Докажите, что не существует многочлена (степени больше нуля) с~целыми
коэффициентами, принимающего при каждом натуральном значении аргумента
значение, равное некоторому простому числу.

\item
Придумайте многочлен с~целыми коэффициентами, корнем которого является число
$\sqrt[5]{2 + \sqrt{3}} + \sqrt[5]{2 - \sqrt{3}}$.

\end{problems}


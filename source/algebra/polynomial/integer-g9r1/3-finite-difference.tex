% $date: 2018-11-06 p2
% $date$groups:
%   g9r1: 2018-11-06 p2

\worksheet
    {Разностное дифференцирование многочлена и последовательности}

% $authors:
% - Александр Савельевич Штерн

\begingroup
    \ifdefined\mupDelta
        \def\diff{\mathup{\mupDelta}}
    \fi \ifdefined\upDelta
        \def\diff{\upDelta}
    \fi
    \providecommand\diff{\Delta}

Для любого многочлена $f(x)$ его разностными производными называются многочлены
\[
    (\diff f)(x) = f(x + 1) - f(x)
\, , \quad
    (\diff^2 f)(x) = (\diff(\diff f))(x)
\, , \quad \ldots
\]

\begin{problems}

\item
Пусть $f(x) = a_{n} x^{n} + a_{n-1} x^{n-1} + \ldots + a_{1} x + a_{0}$.
Найдите $(\diff^{n} f)(x)$.

\item
Существует ли многочлен с целыми коэффициентами степени $n$, которых во всех
точках $i = 0, 1, 2, \ldots, n + 1$ принимает значения $3^{i}$?

\item
Существует ли многочлен с целыми коэффициентами степени $n$, которых во всех
точках $i = 0, 1, 2, \ldots, n$ принимает значения $3^{i}$?
Если существует, представьте его в разумном виде.
\par
\emph{Подсказка: сначала сделайте то же самое для значений $2^{i}$.}

\item
Приведенный многочлен степени $n$ во всех целых точках принимает значение,
кратное натуральному числу $m$.
Докажите, что $n!$ делится на $m$.

\item
Докажите, что для любого многочлена $g(x)$ степени $n$ существует единственный
многочлен $f(x)$ с нулевым свободным членом, такой, что $(\diff f)(x) = g(x)$.

\item
Придумайте разумный вывод формулы суммы $n$ первых кубов, не использующий
готовый ответ.

\item
Докажите, что любое целое число можно представить в виде суммы нескольких
различных точных кубов.

\item
Существует ли квадратный трёхчлен с целыми коэффициентами, значения которого
при всех натуральных значениях аргумента есть степени двойки?

\end{problems}

\endgroup % \def\diff


% $date: 2018-11-05 p3
% $date$groups:
%   g9r2: 2018-11-05 p3

\worksheet{Теорема Виета для квадратных и кубических уравнений}

% $authors:
% - Александр Савельевич Штерн

%В силу теоремы Безу приведённый кубический многочлен с тремя различными корнями
%$a$, $b$, $c$ можно представить в виде
%$x^{3} + m x^{2} + n x + k = (x - a) (x - b) (x - c)$.
%
%Раскрывая в правой части скобки, получаем формулы Виета, по которым
%коэффициенты приведённого кубического многочлена выражаются через его корни.
%
%Верна и обратная теорема Виета, причём для её доказательства даже не нужно
%ссылаться на теорему Безу.
%
%\textit{Упражнение.}
%Три числа $x$, $y$, $z$ подобраны так, что их сумма, произведение и сумма
%попарных произведений положительны.
%Докажите, что каждое число положительно.
%
%Теорему Виета удобно использовать для рассмотрения целых корней многочлена
%с~целыми коэффициентами.
%При этом важно помнить следующий факт: целый корень многочлена с~целыми
%коэффициентами есть делитель его свободного члена.
%
%\textit{Упражнение.}
%Квадратный трёхчлен, у которого все коэффициенты нечётные, не имеет целых
%корней.

%\subsubsection*{Задачи}

\begin{problems}

\item
Существуют ли такие действительные числа $b$ и $c$, что каждое из уравнений
$x^{2} + b x + c = 0$ и $2 x^{2} + (b + 1) x + c + 1 = 0$ имеет по два целых
корня?

\item
Известно, что $x_1$, $x_2$, $x_3$~--- корни
уравнения $x^{3} - 2 x^{2} + x + 1 = 0$.
Составьте новое уравнение, корнями которого были бы числа
$x_1 x_2$, $x_1 x_3$, $x_2 x_3$.

\item
Можно ли подобрать три действительных числа так, чтобы их сумма была равна
числу $a$, сумма попарных произведений была равна числу $a^{2}$, а произведение
равно числу $a^{3}$?

\item
Верно ли, что для любых трёх различных натуральных чисел $a$, $b$, $c$
найдётся квадратный трёхчлен с целыми коэффициентами, принимающий в этих точках
значения $a^3$, $b^3$, $c^3$ соответственно?
\emph{(Всерос-2017, 9.6)}

\item
Числа $a$, $b$, $c$ таковы, что уравнение $x^{3} + a x^{2} + b x + c = 0$ имеет
три корня, и выполнено условие $a + b + c \in [-2, 0]$.
Докажите, что хотя бы один из корней принадлежит отрезку $[0, 2]$.
\emph{(Всерос-2008, 9.2)}

\item
Целые числа $a$, $b$ и $c$ таковы, что числа $a / b + b / c + c / a$
и $a / c + c / b + b / a$ тоже целые.
Докажите, что $|a| = |b| = |c|$.

\item
Миша решил уравнение $x^2 + a x + b = 0$ и сообщил Диме набор из четырёх
чисел~--- два корня и два коэффициента этого уравнения (но не сказал, какие
именно из них корни, а какие~--- коэффициенты).
\\
\subproblem
Сможет ли Дима узнать, какое уравнение решал Миша, если все числа набора оказались различными?
\\
\subproblem
Тот же вопрос, если последнее условие не выполняется.

\item
Сколькими способами числа $2^{0}$, $2^{1}$, $2^{2}$, \ldots, $2^{2005}$ можно
разбить на два непустых множества $A$ и $B$ так, чтобы уравнение
$x^2 - S(A) x + S(B) = 0$, где $S(M)$~--- сумма чисел множества~$M$, имело
целый корень?

\item
Числа от $51$ до $150$ расставлены в таблицу $10 \times 10$.
Может ли случиться, что для каждой пары чисел $a$, $b$, стоящих в соседних по
стороне клетках, хотя бы одно из уравнений
$x^{2} - a x + b = 0$ и $x^{2} - b x + a = 0$ имеет два целых корня?

\item
Даны $n > 1$ приведённых квадратных трёхчленов
$x^{2} - a_{1} x + b_{1}$, \ldots, $x^{2} - a_{n} x + b_{n}$,
причём все $2n$ чисел
$a_{1}$, \ldots, $a_{n}$, $b_{1}$, \ldots, $b_{n}$
различны.
Может ли случиться, что каждое из чисел
$a_{1}$, \ldots, $a_{n}$, $b_{1}$, \ldots, $b_{n}$
является корнем одного из этих трёхчленов?

\end{problems}


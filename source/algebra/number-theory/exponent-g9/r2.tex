% $date: 2018-11-15 p2
% $date$groups:
%   g9r2: 2018-11-15 p2

\worksheet{Про показатели}

% $authors:
% - Леонид Андреевич Попов

\begingroup
    \ifdefined\mupphi
        \def\eulerphi{\mathup{\mupphi}}%
    \fi \ifdefined\upphi
        \def\eulerphi{\upphi}%
    \fi
    \providecommand\eulerphi{\phi}%

\begin{problems}

\item
Найдите все простые $p$ и $q$ такие, что $q \mid 2^{p} - 1$
и~$p \mid 2^{q} - 1$.

\item
Докажите, что если $a > 1$, то~$n$ делит $\eulerphi(a^{n} - 1)$.

\item
Пусть $p > 2$~-- простое число.
Докажите, что любой простой делитель числа $(a^{p} - 1)$ или делит $(a - 1)$
или имеет вид $2 p x + 1$.

\item
Докажите, что любой нечетный простой делитель числа $a^{2^{k}} + 1$ имеет вид
$2^{k+1} x + 1$.

\item
Пусть $p$~--- простое число.
Докажите, что все простые делители числа $p^{p} - 1$ и большие $p$ дают
остаток~1 при делении на $p$.

\item
Пусть $p$ и~$q$~--- простые числа, большие $5$.
Докажите, что если $p \mid 2^{q} + 3^{q}$, то~$q < p$.

\item
Пусть $p$~--- простое число, $d$~--- один из делителей числа $p - 1$.
Выберем из остатков $1, 2, \ldots, p - 1$ те, чей показатель по модулю~$p$
равен $d$.
Чему равен остаток произведения всех выбранных чисел по модулю~$p$?

\item
Найдите все простые $p$ и $q$, для которых $5^{p} + 5^{q}$ делится на $p q$.
% http://artofproblemsolving.com/community/c6t45309f6h559422

\end{problems}

\endgroup % \def\eulerphi


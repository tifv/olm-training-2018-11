% $date: 2018-11-12 p1
% $date$groups:
%   g9r1: 2018-11-12 p1

\worksheet{Показательный разнобой по ТЧ}

% $authors:
% - Леонид Андреевич Попов

\begin{problems}

\item
Пусть $p$~--- простое число, $d$~--- один из делителей числа $p - 1$.
Выберем из остатков $1, 2, \ldots, p - 1$ те, чей показатель по модулю~$p$
равен $d$.
Чему равен остаток произведения всех выбранных чисел по модулю~$p$?

\item
Пусть $n$~--- чётное число, а натуральные числа $a$ и $b$ взаимно просты.
При каких $a$ и $b$ число $\frac{a^{n} + b^{n}}{a + b}$~-- целое?
% 1.3.4 Number Theory: Structures, Examples, and Problems by Titu Andreescu

\item
Пусть $n$~--- чётное число.
Докажите, что любой делитель $n^4 + 1$ дает остаток $1$ при делении на $8$.

%\item
%Найдите все тройки простых чисел $p$, $q$, $r$ таких, что
%$q^r + 1 \kratno p$, $r^p + 1 \kratno q$, $p^q + 1 \kratno r$.

\item
Докажите, что для любого натурального числа $n$ существует натуральное число
$K$ такое, что сумма цифр $K$ равна $n$, и $K$ делится на $n$.

\item
При каком наименьшем $k$ существуют натуральные числа
$x_{1}$, $x_{2}$, \ldots, $x_{k}$ такие, что
\[
    x_{1}^3 + x_{2}^3 + \ldots + x_{k}^3 = 2002^{2002}
\, ? \]

\item
Найдите все простые $p$ и $q$, для которых $5^{p} + 5^{q}$ делится на $p q$.
% http://artofproblemsolving.com/community/c6t45309f6h559422

\item
Задана последовательность $a_{n} = b^{n+1} + b^{n} - 1$,
где $b \in \mathbb{N}$, $n = 1, 2, 3, \ldots$\,.
Всегда ли из неё можно выбрать $2018$ попарно взаимно простых чисел?

\end{problems}


% $date: 2018-11-14 p3
% $date$groups:
%   g9r2: 2018-11-14 p3

\worksheet{Добавка (теория чисел)}

% $authors:
% - Леонид Андреевич Попов

\begin{problems}

\item
Пусть $n$~--- чётное число, а натуральные числа $a$ и $b$ взаимно просты.
При каких $a$ и $b$ число $\frac{a^{n} + b^{n}}{a + b}$~-- целое?
% 1.3.4 Number Theory: Structures, Examples, and Problems by Titu Andreescu

\item
Докажите, что существует бесконечно много $n$ таких, что уравнение
\[ a^2 + b^2 + 1 = 3^{n} \]
имеет хотя бы одно решение в натуральных числах.
% http://artofproblemsolving.com/community/c6t177f6h1160686

\item
При каком наименьшем $k$ существуют натуральные числа
$x_{1}$, $x_{2}$, \ldots, $x_{k}$ такие, что
\[
    x_{1}^3 + x_{2}^3 + \ldots + x_{k}^3 = 2002^{2002}
\, ? \]

\item
Найдите все тройки простых $p$, $q$, $r$ таких, что
$p \mid q^{r} + 1$,\enspace
$q \mid r^{p} + 1$,\enspace
$r \mid p^{q} + 1$.

\end{problems}

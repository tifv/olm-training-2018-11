% $date: 2018-11-12 p2
% $date$groups:
%   g9r2: 2018-11-12 p2

\worksheet{Сравнения и классические теоремы}

% $authors:
% - Леонид Андреевич Попов

\begingroup
    \ifdefined\mupphi
        \def\eulerphi{\mathup{\mupphi}}%
    \fi \ifdefined\upphi
        \def\eulerphi{\upphi}%
    \fi
    \providecommand\eulerphi{\phi}%

\begin{claim}{Теорема Эйлера}
Пусть $n$~--- натуральное число, $a$~--- взаимно просто с~$n$.
Тогда
\[
    a^{\eulerphi(n)} - 1 \kratno n
\, , \]
где $\eulerphi(n)$~--- функция Эйлера: количество чисел, взаимно простых с~$n$
и~не превосходящих $n$.
\end{claim}

\begin{problems}

\item
Докажите, что число $40^{81} + 17^{160}$ является составным.

\item
Докажите, что любой нечетный простой делитель $a^2 + 1$, где $a$~---
натуральное, имеет вид $4 m + 1$.

\item
Найдите все простые $p$ и натуральные $n$ такие, что $7^{p^{n}} + 1$ делится
на~$p^{n}$.

\item
Докажите, что ни при каком целом $k$ число $k^2 + k + 1$ не делится на $101$.

\item
Пусть $a$~--- нечетное число.
Докажите, что числа $a^{2^{n}} + 2^{2^{n}}$ и $a^{2^{m}} + 2^{2^{m}}$ взаимно
просты при любых натуральных $n \neq m$.

\item
Найдите все натуральные $n$, для которых $n^{n} + 1$ и $(2n)^{2n} + 1$ являются
простыми.
% http://artofproblemsolving.com/community/c6h381151p2110492
% задача-подстава, делается алгебраически

\item
Докажите, что для любого натурального числа $n$ существует натуральное
число~$K$ такое, что сумма цифр $K$ равна $n$, и $K$ делится на $n$.

\item
Даны натуральные числа $k$ и $n$.
Артём пишет на блокноте числа $n$, $n^{n}$, $n^{n^{n}}$, $n^{n^{n^{n}}}$
и так далее (каждое новое число получается возведением числа~$n$ в~степень
предыдущего числа).
Докажите, что рано или поздно остатки при делении на $k$ у чисел Артёма будут
одни и те же.


\end{problems}

\endgroup % \def\eulerphi


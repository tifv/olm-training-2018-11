% $date: 2018-11-15 p1
% $date$groups:
%   g9r1: 2018-11-15 p1

\worksheet{Конструктивный разнобой}

% $authors:
% - Леонид Андреевич Попов

\begin{problems}

\item
Существуют ли $2018$ натуральных чисел, образующих арифметическую прогрессию,
таких, что каждое имеет вид $a^{b}$, где $a$, $b$~--- натуральные и $b > 1$?

\item
Даны натуральные числа $k$ и $n$.
Артём пишет на блокноте числа $n$, $n^{n}$, $n^{n^{n}}$, $n^{n^{n^{n}}}$
и так далее (каждое новое число получается возведением числа~$n$ в~степень
предыдущего числа).
Докажите, что рано или поздно остатки при делении на $k$ у чисел Артёма будут
одни и те же.

\item
Существуют ли $100$ натуральных чисел таких, что НОД любых двух из них равен их
разности?

\item
Задана последовательность $a_{n} = 2^{n} - 3$, где $n = 1, 2, 3, \ldots$\,.
Можно ли выбрать из неё $2018$ попарно взаимно простых чисел?

\item
Найдите наименьший простой делитель числа $12^{2^{15}} + 1$.

\item
Существует ли натуральное $N$ такое, что у него ровно $2000$ простых делителей
и $2^{N} + 1$ делится на $N$?

\item
Пусть $k = 2^{2^{n}} + 1$ для некоторого натурального $n$.
Докажите, что $k$ является простым тогда и только тогда, когда $k$~--- делитель
числа $3^{\frac{k-1}{2}} + 1$.
% Андрееску 9.1.1

\end{problems}


% $date: 2018-11-10

% $document$style[print]:
% - .[a6paper,resize-to]

\worksheet{Числа Пизо}

% $authors:
% - Иван Викторович Митрофанов

Возможно, вы видели такую задачу: у~числа $(1 + \sqrt{2})^{1001}$ найти в~уме
первые пять цифр после запятой.
Ответ~--- это ${,}00000{\ldots}$, то~есть большая степень иррационального числа
оказывается очень близкой к~целому числу.

С~помощью многочленов мы поймем, у~каких еще иррациональных чисел все большие
степени <<почти целые>>, и~поговорим про гипотезу Пизо-Виджаярагхавана.


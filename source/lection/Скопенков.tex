% $date: 2018-11-10

% $build$style[print]:
% - .[a6paper,resize-to]

\worksheet*{Рамсеевская теория зацеплений\\ и~реализуемость гиперграфов}

% $authors:
% - Аркадий Борисович Скопенков

Можно~ли граф расположить на~плоскости так, чтобы его ребра не~пересекались
и~не~самопересекались?
Мы рассмотрим аналогичную задачу о~реализуемости \emph{гиперграфов}
в~трехмерном и~четырехмерном пространствах.
Теория гиперграфов~--- бурно развивающийся раздел математики, возникший
на~стыке комбинаторики, топологии и~программирования.

Результаты о~гиперграфах будут обсуждаться на~элементарном языке систем точек.
Мы разберем метод понижения размерности, который сделает доступным
доказательства для четырехмерного пространства и~пригодится Вам при решении
олимпиадных задач.
За~счет этого для понимания лекции
 не~требуется предварительных знаний по~гиперграфам;
 достаточно начального знакомства со~стереометрией.
Решение следующей задачи поможет Вам оценить, будет~ли лекция доступнa для Вас.

\smallskip
\begin{em}
Cреди любых~ли шести точек в~пространстве можно выбрать пять точек $O,A,B,C,D$,
для которых (двумерные) треугольники $OAB$  и~$OBC$ имеют общую точку, отличную
от~$O$?
\end{em}



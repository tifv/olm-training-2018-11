% $date: 2018-11-07
% $timetable:
%   g9r1:
%     2018-11-07:
%       2:

\worksheet*{Инверсия}

% $authors:
% - Фёдор Львович Бахарев

\begin{problems}

\item
\subproblem
Пусть при инверсии с~центром~$O$ точка $A$ переходит в~$A'$, а~точка~$B$
переходит в~$B'$.
Докажите, что треугольники $OAB$ и~$OB'A'$ подобны.
\\
\subproblem
Докажите, что при инверсии с~центром~$O$ прямая~$\ell$, не~проходящая
через $O$, переходит в~окружность, проходящую через $O$, а~окружность,
проходящая через $O$, переходит в~прямую, не~проходящую через $O$.
\\
\subproblem
Докажите, что при инверсии с~центром $O$ окружность, не~проходящая через $O$,
переходит в~окружность, не~проходящую через $O$.

\item
\subproblem
Докажите, что при инверсии касающиеся окружности (прямая и~окружность)
переходят в~касающиеся окружности, или в~касающиеся окружность и~прямую,
или в~пару параллельных прямых.
\\
\subproblem
Докажите, что при инверсии сохраняется угол между окружностями
(между окружностью и~прямой, между прямыми).

\item
Точки $A$ и~$B$ лежат на~окружности~$\omega$.
Касательные к~окружности, проходящие через точки $A$ и~$B$, пересекаются
в~точке~$P$.
Докажите, что $P$ является образом середины хорды~$AB$ при инверсии
относительно $\omega$.

\item
\subproblem
Четырехугольник $ABCD$ вписан в~окружность, $AB \cdot CD = AD \cdot BC$.
При инверсии с~центром в~точке~$A$ и~радиусом~$1$
точка~$B$ переходит в~$B'$,
$C$~--- в~$C'$
и~$D$~--- в~$D'$.
Докажите, что $B'C' = C'D'$.
\\
\subproblem
Докажите с~помощью инверсии теорему Птолемея.

\item
Окружности $\Gamma_{1}$ и~$\Gamma_{3}$ касаются внешним образом в~точке~$P$,
и~окружности $\Gamma_{2}$ и~$\Gamma_{4}$ тоже касаются внешним образом
в~точке~$P$.
При этом окружности
$\Gamma_{1}$ и~$\Gamma_{2}$ повторно пересекаются в~точке~$A$,
$\Gamma_{2}$ и~$\Gamma_{3}$~--- в~точке~$B$,
$\Gamma_{3}$ и~$\Gamma_{4}$~--- в~точке~$C$,
$\Gamma_{4}$ и~$\Gamma_{1}$~--- в~точке~$D$.
Докажите, что
\[
    \frac{AB \cdot BC}{CD \cdot DA} = \frac{PB^2}{PD^2}
\, . \]

\item
Пусть $P$~--- точка внутри треугольника $ABC$ такая, что
\[
    \angle APB - \angle ACB = \angle APC - \angle ABC
\, . \]
Пусть точки $D$ и~$E$~--- центры вписанных окружностей
треугольников $APB$ и~$APC$ соответственно.
Докажите, что прямые $AP$, $BD$ и~$CE$ пересекаются в~одной точке.

\item
Окружность с~центром~$O$ проходит через вершины $A$ и~$C$ треугольника $ABC$
и~пересекает стороны $AB$ и~$BC$ повторно в~точках $K$ и~$N$ соответственно.
Пусть $M$~--- точка пересечения описанных окружностей треугольников $ABC$
и~$KBN$ (отличная от~$B$).
Докажите, что $\angle OMB = 90^{\circ}$.

\item
Пусть $p$~--- полупериметр треугольника $ABC$.
Точки $E$ и~$F$ на~прямой~$BC$ таковы, что $AE = AF = p$.
Докажите, что описанная окружность треугольника $AEF$ касается вневписанной
окружности треугольника $ABC$ со~стороны~$BC$.

\item
Из~точки~$K$ проведены касательные $KN$ и~$KL$ к~окружности~$\omega$.
На~луче~$KN$ за~точкой~$N$ выбрана точка~$M$.
Описанная окружность треугольника $KLM$ повторно пересекает окружность~$\omega$
в~точке $P$, а~$Q$~--- проекция точки~$N$ на~$ML$.
Докажите, что $\angle MPQ = 2 \angle KML$.

\item
Вписанная окружность~$\omega$ треугольника $ABC$ касается стороны~$BC$
в~точке~$K$.
Прямая, проходящая через $K$ и~середину высоты~$AD$, повторно пересекает
$\omega$ в~точке~$N$.
Докажите, что описанная окружность треугольника $BCN$ касается $\omega$.

\end{problems}


% $date: 2018-11-09 p2
% $date$groups:
%   g9r2: 2018-11-09 p2

\worksheet{Еще задачи на~инверсию}

% $authors:
% - Фёдор Львович Бахарев

\begin{problems}

\item
Через точку~$A$ к~окружности~$\omega$ с~центром~$O$ проведены
касательные $AX$ и~$AY$, а~также секущая, пересекающая окружность
в~точках $Z$ и~$T$.
Докажите, что точки $Z$, $T$, $O$ и~середина~$XY$ лежат на~одной окружности.

\item\emph{С~помощью инверсии!}\\
\subproblem
Окружность~$s_{1}$ касается окружности~$s$ внутренним образом в~точке~$N$.
Хорда~$AB$ окружности~$s$ касается окружности~$s_{1}$ в~точке~$M$.
Докажите, что $MN$ делит дугу~$AB$, не~содержащую точку~$N$, пополам.
\\
\subproblem
Окружности $s_{1}$ и~$s_{2}$ касаются окружности~$s$ внутренним образом.
Хорда~$AB$ окружности~$s$ является общей внешней касательной
окружностей $s_{1}$ и~$s_{2}$.
Докажите, что их радикальная ось проходит через середину дуги~$AB$.

\item
Пусть $p$~--- полупериметр треугольника $ABC$.
Точки $E$ и~$F$ на~прямой $BC$ таковы, что $AE = AF = p$.
Докажите, что описанная окружность треугольника $AEF$ касается вневписанной
окружности треугольника $ABC$ со~стороны~$BC$.

\item
Окружности $\Gamma_{1}$ и~$\Gamma_{3}$ касаются внешним образом в~точке~$P$,
и~окружности $\Gamma_{2}$ и~$\Gamma_{4}$ тоже касаются внешним образом
в~точке~$P$.
При этом окружности
$\Gamma_{1}$ и~$\Gamma_{2}$ повторно пересекаются в~точке~$A$,
$\Gamma_{2}$ и~$\Gamma_{3}$~--- в~точке~$B$,
$\Gamma_{3}$ и~$\Gamma_{4}$~--- в~точке~$C$,
$\Gamma_{4}$ и~$\Gamma_{1}$~--- в~точке~$D$.
Докажите, что
\[
    \frac{AB \cdot BC}{CD \cdot DA} = \frac{PB^2}{PD^2}
\, . \]

\item
Пусть $P$~--- точка внутри треугольника $ABC$ такая, что
\[
    \angle APB - \angle ACB = \angle APC - \angle ABC
\, . \]
Пусть точки $D$ и~$E$~--- центры вписанных окружностей
треугольников $APB$ и~$APC$ соответственно.
Докажите, что прямые $AP$, $BD$ и~$CE$ пересекаются в~одной точке.

\item
В~полуокружность~$\omega$, стягиваемую диаметром~$PQ$, вписана окружность,
касающаяся диаметра в~точке~$C$.
Точки $A$ на~$\omega$ и~$B$ на~отрезке~$CQ$ таковы, что $AB$ касается
окружности и~$AB \perp PQ$.
Докажите, что $CA$~--- биссектриса угла $BAP$.

\item
Из~точки~$K$ проведены касательные $KN$ и~$KL$ к~окружности~$\omega$.
На~луче~$KN$ за~точкой~$N$ выбрана точка~$M$.
Описанная окружность треугольника $KLM$ повторно пересекает окружность~$\omega$
в~точке $P$, а~$Q$~--- проекция точки~$N$ на~$ML$.
Докажите, что $\angle MPQ = 2 \angle KML$.

\item
На~прямой~$\ell$ заданы точки $A$, $B$ и~$C$ в~указанном порядке.
На~отрезках $AB$, $BC$ и~$CA$ по~одну сторону от~прямой~$\ell$ построены
полуокружности $\omega_{1}$, $\omega_{2}$ и~$\omega$ соответственно.
Окружность~$\gamma_{1}$ касается всех трех полуокружностей.
Окружности $\gamma_{n}$ при $n \geq 2$ касаются
полуокружностей $\omega$ и~$\omega_{1}$ и~окружности $\gamma_{n-1}$.
Найдите отношение расстояния от~центра окружности~$\gamma_{n}$ до~прямой~$\ell$
к~ее радиусу.

\end{problems}


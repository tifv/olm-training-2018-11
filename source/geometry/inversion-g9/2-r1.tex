% $date: 2018-11-09
% $timetable:
%   g9r1:
%     2018-11-09:
%       1:

\worksheet*{Инверсия еще}

% $authors:
% - Фёдор Львович Бахарев

\begin{problems}

\item
Через точку~$A$ к~окружности~$\omega$ с~центром~$O$ проведены
касательные $AX$ и~$AY$, а~также секущая, пересекающая окружность
в~точках $Z$ и~$T$.
Докажите, что точки $Z$, $T$, $O$ и~середина~$XY$ лежат на~одной окружности.

\item
В~остроугольном треугольнике $ABC$ проведены высоты $AA_{1}$ и~$BB_{1}$.
Прямая~$A_{1}B_{1}$ пересекает описанную окружность треугольника $ABC$
в~точках $X$ и~$Y$.
Окружность, описанная около треугольника $CBB_{1}$, пересекает высоту~$AA_{1}$
в~точке~$Z$.
Докажите, что $CX = CY = CZ$.

\item
В~полуокружность~$\omega$, стягиваемую диаметром~$PQ$, вписана окружность,
касающаяся диаметра в~точке~$C$.
Точки $A$ на~$\omega$ и~$B$ на~отрезке~$CQ$ таковы, что $AB$ касается
окружности и~$AB \perp PQ$.
Докажите, что $CA$~--- биссектриса угла $BAP$.

\item\emph{С~помощью инверсии!}\\
\subproblem
Окружность~$s_{1}$ касается окружности~$s$ внутренним образом в~точке~$N$.
Хорда~$AB$ окружности~$s$ касается окружности~$s_{1}$ в~точке~$M$.
Докажите, что $MN$ делит дугу~$AB$, не~содержащую точку~$N$, пополам.
\\
\subproblem
Окружности $s_{1}$ и~$s_{2}$ касаются окружности~$s$ внутренним образом.
Хорда~$AB$ окружности~$s$ является общей внешней касательной
окружностей $s_{1}$ и~$s_{2}$.
Докажите, что их радикальная ось проходит через середину дуги~$AB$.

\item
На~прямой~$\ell$ заданы точки $A$, $B$ и~$C$ в~указанном порядке.
На~отрезках $AB$, $BC$ и~$CA$ по~одну сторону от~прямой~$\ell$ построены
полуокружности $\omega_{1}$, $\omega_{2}$ и~$\omega$ соответственно.
Окружность~$\gamma_{1}$ касается всех трех полуокружностей.
Окружности $\gamma_{n}$ при $n \geq 2$ касаются
полуокружностей $\omega$ и~$\omega_{1}$ и~окружности $\gamma_{n-1}$.
Найдите отношение расстояния от~центра окружности~$\gamma_{n}$ до~прямой~$\ell$
к~ее радиусу.

\item
Точка~$I$~--- центр вписанной окружности треугольника $ABC$.
Меньшую из~окружностей, проходящих через точку~$I$ и~касающихся
сторон $AB$ и~$AC$, обозначим через $\omega_{A}$.
Аналогично определим $\omega_{B}$ и~$\omega_{C}$.
Вторые точки пересечения
$\omega_{A}$ и~$\omega_{B}$,
$\omega_{B}$ и~$\omega_{C}$,
$\omega_{C}$ и~$\omega_{A}$
обозначим через $C_{1}$, $A_{1}$, $B_{1}$ соответственно.
Докажите, что центры окружностей, описанных около
треугольников $AA_{1}I$, $BB_{1}I$ и~$CC_{1}I$, лежат на~одной прямой.

\item
В~треугольнике $ABC$ проведена биссектриса~$AD$.
Точки $M$ и~$N$ являются проекциями вершин $B$ и~$C$ на~$AD$.
Окружность с~диаметром~$MN$ пересекает $BC$ в~точках $X$ и~$Y$.
Докажите, что $\angle BAX = \angle CAY$.
% http://problems.ru/view_problem_details_new.php?id=64473

\item
Точка~$N$~--- середина дуги $BAC$ описанной около треугольника $ABC$
окружности~$\omega$.
Касательные в~точках $A$ и~$N$ к~окружности~$\omega$ пересекаются в~точке~$X$.
Касательные в~точках $B$ и~$C$~--- в~точке~$Y$.
Докажите, что прямая~$XY$ проходит через основание биссектрисы угла~$A$.

\item
Четырёхугольник $ABCD$ описан около окружности с~центром~$I$.
Касательные к~описанной окружности треугольника $AIC$ в~точках $A$, $C$
пересекаются в~точке~$X$.
Касательные к~описанной окружности треугольника $BID$ в~точках $B$, $D$
пересекаются в~точке~$Y$.
Докажите, что точки $X$, $I$, $Y$ лежат на~одной прямой.
% http://problems.ru/view_problem_details_new.php?id=64881

\end{problems}


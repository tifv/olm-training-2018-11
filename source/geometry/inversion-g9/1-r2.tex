% $date: 2018-11-07
% $timetable:
%   g9r2:
%     2018-11-07:
%       3:

\worksheet*{Введение в инверсию}

% $authors:
% - Фёдор Львович Бахарев

\begin{problems}

\item
\subproblem
Пусть при инверсии с~центром~$O$ точка $A$ переходит в~$A'$, а~точка~$B$
переходит в~$B'$.
Докажите, что треугольники $OAB$ и~$OB'A'$ подобны.
\\
\subproblem
Докажите, что при инверсии с~центром~$O$ прямая~$\ell$, не~проходящая
через $O$, переходит в~окружность, проходящую через $O$, а~окружность,
проходящая через $O$, переходит в~прямую, не~проходящую через $O$.
\\
\subproblem
Докажите, что при инверсии с~центром $O$ окружность, не~проходящая через $O$,
переходит в~окружность, не~проходящую через $O$.

\item
\subproblem
Докажите, что при инверсии касающиеся окружности (прямая и~окружность)
переходят в~касающиеся окружности, или в~касающиеся окружность и~прямую,
или в~пару параллельных прямых.
\\
\subproblem
Докажите, что при инверсии сохраняется угол между окружностями
(между окружностью и~прямой, между прямыми).

\item
Точки $A$ и~$B$ лежат на~окружности~$\omega$.
Касательные к~окружности, проходящие через точки $A$ и~$B$, пересекаются
в~точке~$P$.
Докажите, что $P$ является образом середины хорды~$AB$ при инверсии
относительно $\omega$.

\item
\subproblem
Четырехугольник $ABCD$ вписан в~окружность, $AB \cdot CD = AD \cdot BC$.
При инверсии с~центром в~точке~$A$ и~радиусом~$1$
точка~$B$ переходит в~$B'$,
$C$~--- в~$C'$
и~$D$~--- в~$D'$.
Докажите, что $B'C' = C'D'$.
\\
\subproblem
Докажите с~помощью инверсии теорему Птолемея.

\item
Что является образом описанной окружности треугольника при инверсии
относительно вписанной окружности?

\item
Дана окружность и~точка~$P$ внутри нее, отличная от~центра.
Рассматриваются пары окружностей, касающиеся данной изнутри и~друг друга
в~точке~$P$.
Найдите геометрическое место точек пересечения общих внешних касательных к~этим
окружностям.

\item
Докажите, что две непересекающиеся окружности $S_{1}$ и~$S_{2}$ можно при
помощи инверсии перевести в~пару концентрических окружностей.

\item
В~остроугольном треугольнике $ABC$ проведены высоты $AA_{1}$ и~$BB_{1}$.
\\
\subproblem
Что является образом прямой $A_{1}B_{1}$ при инверсии с~центром~$C$, которая
переводит точку~$A_{1}$ в~точку~$B$?
\\
\subproblem
Прямая $A_{1}B_{1}$ пересекает описанную окружность треугольника $ABC$
в~точках $X$ и~$Y$.
Окружность, описанная около треугольника $CBB_{1}$, пересекает высоту~$AA_{1}$
в~точке~$Z$.
Докажите, что $CX = CY = CZ$. 

\end{problems}


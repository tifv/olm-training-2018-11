% $date: 2018-11-07 p1
% $date$groups:
%   gX: 2018-11-07 p1

\worksheet{Ближневосточный разнобой}

% $authors:
% - Фёдор Львович Бахарев

\begin{problems}

\item
К~двум пересекающимся окружностям $\omega_{1}$ и~$\omega_{2}$ проведены общие
внешние касательные $AB$ и~$CD$ так, что
$A, C \in \omega_{1}$ и~$B, D \in \omega_{2}$.
Касательные к~окружностям, проходящие через середину~$M$ отрезка~$AB$
и~отличные от~$AB$, пересекают прямую~$CD$ в~точках $K$ и~$L$.
Докажите, что центр вписанной окружности треугольника $MKL$ равноудален
от~точек $C$ и~$D$.
% https://artofproblemsolving.com/community/c6h1701132p10928109

\item
Пусть $H$~--- ортоцентр остроугольного треугольника $ABC$, а~$\omega$~---
окружность девяти точек.
На~дуге $BHC$ окружности~$\Omega$ с~центром~$O'$ выбрана точка~$X$.
Отрезок~$AX$ пересекает окружность~$\omega$ в~точке~$Y$.
Точка~$P$ на~окружности~$\omega$ такова, что $PX = PY$.
Докажите, что $\angle O'PX = 90^\circ$.
% https://artofproblemsolving.com/community/c6h1701140p10928167

\item
Высоты $BE$ и~$CF$ остроугольного треугольника $ABC$ пересекаются в~точке~$H$.
Точка~$P$ на~$EF$ такова, что $HO \perp HP$
($O$~--- центр описанной окружности треугольника $ABC$),
а~точка~$Q$ на~$AH$ такова, что $HM \perp PQ$ ($M$~--- середина~$BC$).
Докажите, что $QA = 3 QH$.
% https://artofproblemsolving.com/community/c6h1701020p10927044

\item
В~треугольнике $ABC$ точка~$M$~--- середина~$BC$.
Пусть $\omega$~--- окружность, лежащая внутри треугольника $ABC$, касающаяся
сторон $AB$ и~$AC$ в~точках $E$ и~$F$ соответственно.
Из~точки~$M$ проведены касательные $MP$ и~$MQ$ к~$\omega$ так, что
точки $P$ и~$B$ лежат по~одну сторону от~прямой~$AM$.
Пусть $X = PM \cap BF $ и~$Y = QM \cap CE $.
Докажите, что если $2 PM = BC$, то~прямая~$XY$ касается $\omega$.
% https://artofproblemsolving.com/community/c6h1623012p10163453

\item
Биссектриса угла~$A$ треугольника $ABC$ пересекает высоты $BE$ и~$CF$
в~точках $M$ и~$N$ соответственно.
Докажите, что перпендикуляры из~$M$ на~$EF$ и~из~$N$ на~$BC$, а~также медиана
из~вершины~$A$ пересекаются в~одной точке.
% https://artofproblemsolving.com/community/c6h1623417p10167655

\item
Точки~$P$ на~описанной окружности~$\omega$ и~$Q$ на~основании~$BC$
равнобедренного треугольника $ABC$ таковы, что $AP = AQ$.
При этом прямые $AP$ и~$BC$ пересекаются в~точке~$R$.
Докажите, что касательные из~точек $B$ и~$C$ к~вписанной окружности
треугольника $AQR$, отличные от~$BC$, пересекаются на~$\omega$.
% https://artofproblemsolving.com/community/c6h1628676p10217476

\item
Треугольник $ABC$ таков, что
$\angle ABC < \angle BCA < \angle CAB < 90^{\circ}$.
Точка~$O$~--- центр его описанной окружности, а~точка~$K$ симметрична $O$
относительно $BC$.
Точки $D$ и~$E$~--- основания перпендикуляров из~точки~$K$
на~прямые $AB$ и~$AC$, соответственно.
Прямая~$DE$ пересекает $BC$ в~точке~$P$, а~окружность~$\omega$ с~диаметром~$AK$
пересекает описанную окружность треугольника $ABC$ в~точке~$Q$, отличной
от~$A$.
Докажите, что $PQ$ пересекает серединный перпендикуляр к~$BC$ в~точке
на~окружности~$\omega$.
% https://artofproblemsolving.com/community/c6h1614072p10086315

\item
Точка~$H'$ симметрична основанию высоты~$AH$ треугольника $ABC$ относительно
середины стороны~$BC$.
Касательные к~описанной окружности треугольника $ABC$ в~точках $B$ и~$C$
пересекаются в~точке~$X$.
Прямая, проходящая через $H'$ и~перпендикулярная $XH'$ пересекает
прямые $AB$ и~$AC$ в~точках $Y$ и~$Z$ соответственно.
Докажите, что $\angle YXB = \angle ZXC$.
% https://artofproblemsolving.com/community/c6h1095220p4902494

\end{problems}


% $date: 2018-11-12
% $timetable:
%   gX:
%     2018-11-12:
%       1:

\worksheet*{Дальневосточный разнобой}

% $authors:
% - Фёдор Львович Бахарев

\begin{problems}

\item
Точка~$A'$ диаметрально противоположна вершине~$A$ на~описанной окружности
треугольника $ABC$.
На~стороне~$BC$ во~внешнюю сторону построен равносторонний треугольник $BCD$.
Перпендикуляр к~$A'D$, проведенный в~точке~$A'$, пересекает прямые $CA$ и~$AB$
в~точках $E$ и~$F$ соответственно.
На~отрезке~$EF$ построен равнобедренный треугольник $ETF$
с~$\angle ETF = 120^\circ$ (точки $A$ и~$T$ по~разные стороны от~$EF$).
Докажите, что $AT$ проходит через центр окружности девяти точек
треугольника $ABC$.
% https://artofproblemsolving.com/community/c6h1437592p8154775

\item
Треугольник $ABC$ с~инцентром~$I$ вписан в~окружность~$\Gamma$.
Точка~$M$~--- середина стороны~$BC$.
Точки $D$, $E$ и~$F$ выбраны на~сторонах $BC$, $CA$ и~$AB$ соответственно так,
что $ID \perp BC$, $IE\perp AI$ и~$IF \perp AI$.
Предположим, что описанная окружность треугольника $AEF$ повторно пересекает
$\Gamma$ в~точке~$X$.
Докажите, что прямые $XD$ и~$AM$ пересекаются на~$\Gamma$.
% https://artofproblemsolving.com/community/c6h1480699p8639270

\item
В~неравнобедренном треугольнике $ABC$ точки $D$, $F$, и~$G$~--- середины
сторон $AB$, $AC$ и~$BC$ соответственно.
Окружность~$\Gamma$ проходит через $C$, касается $AB$ в~точке~$D$
и~пересекает $AF$ и~$BG$ в~точках $X$ и~$Y$ соответственно.
Точки $X'$ и~$Y'$ симметричны точкам $X$ и~$Y$ относительно $F$ и~$G$
соответственно.
Прямая~$X'Y'$ пересекает $CD$ и~$FG$ в~точках $Q$ и~$M$ соответственно.
Наконец, прямая $CM$ повторно пересекает $\Gamma$ в~точке~$P$.
Докажите, что $CQ = QP$.
% https://artofproblemsolving.com/community/c6h1268908p6622796

\item
На~сторонах $AB$, $BC$, $CD$ и~$DA$ выпуклого четырехугольника $ABCD$ выбраны
точки $P$, $Q$, $R$ и~$S$ соответственно.
Отрезки $PR$ и~$QS$ пересекаются в~точке~$O$.
Предположим, что четырехугольники $APOS$, $BQOP$, $CROQ$ и~$DSOR$ являются
описанными.
Докажите, что прямые $AC$, $PQ$ и~$RS$ пересекаются в~одной точке или
параллельны.
% https://artofproblemsolving.com/community/c6h1268837p6622082

\item
Окружность~$\Gamma$ проходит через вершину~$A$ треугольника $ABC$
и~пересекает стороны $AB$ и~$AC$ в~точках $E$ и~$F$ соответственно,
а~описанную окружность треугольника $ABC$ вторично в~точке~$P$.
Докажите, что точка, симметричная $P$ относительно $EF$, лежит на~$BC$
тогда и~только тогда, когда $\Gamma$ проходит через центр описанной
окружности треугольника $ABC$.

\item
В~остроугольном треугольнике $ABC$ точка~$O$~--- центр описанной окружности,
$G$~--- центр масс.
Пусть $D$~--- середина стороны~$BC$, $E$~--- точка на~окружности
с~диаметром~$BC$, лежащая внутри треугольника $ABC$, такая, что $AE \perp BC$.
Пусть $F = EG \cap OD$ и~точки $K$ и~$L$ на~прямой~$BC$ таковы, что
$FK \parallel OB$ и~$FL \parallel OC$.
Кроме того, точка $M \in AB$ такова, что $MK \perp BC$
и~точка $N \in AC$ такова, что $NL \perp BC$.
Наконец, окружность~$\omega$ касается отрезков $OB$ и~$OC$ в~точках $B$ и~$C$.
Если вы дочитали условие до~конца, то~докажите, что описанная окружность
треугольника $AMN$ касается окружности~$\omega$.

\item
Внутри треугольника $ABC$ дана точка~$P$.
Предположим, что $L$, $M$ и~$N$~--- середины сторон $BC$, $CA$ и~$AB$
соответственно и~$PL : PM : PN = BC : CA : AB$.
Продолжения $AP$, $BP$ и~$CP$ пересекают описанную окружность треугольника
$ABC$ в~точках $D$, $E$ и~$F$ соответственно.
Докажите, что центры описанных окружностей
треугольников $APF$, $APE$, $BPF$, $BPD$, $CPD$, $CPE$ лежат на~одной
окружности.

\item
Вписанная окружность неравнобедренного треугольника $ABC$ касается его
сторон $BC$, $AC$, $AB$ в~точках $D$, $E$ и~$F$ соответственно.
Пусть точки $L$, $M$ и~$N$ симметричны точкам $D$, $E$ и~$F$ соответственно
относительно $EF$, $FD$ и~$DE$.
Пусть $AL \cap BC = P$, $BM \cap CA = Q$ и~$CN \cap AB = R$.
Докажите, что точки $P$, $Q$ и~$R$ коллинеарны.

\end{problems}


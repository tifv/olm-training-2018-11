% $date: 2018-11-08 p1
% $date$groups:
%   gX: 2018-11-08 p1

\worksheet{Разнобой для жителей крайнего севера}

% $authors:
% - Фёдор Львович Бахарев

\begin{problems}

\item
Четырехугольник $ABCD$ описан около окружности с~центром в~точке~$O$.
Прямые $AB$ и~$CD$ пересекаются в~точке~$P$, а~прямые $AD$ и~$BC$~---
в~точке~$Q$, причем отрезки $BP$ и~$DQ$ пересекаются в~точке~$A$;
$M$~--- основание перпендикуляра, опущенного из~точки~$O$ на~$PQ$.
Докажите, что
\\
\subproblem
точка пересечения диагоналей четырехугольника $ABCD$ лежит на~прямой~$OM$;
\\
\subproblem
$\angle BMO = \angle DMO$.

\item
Вписанная окружность треугольника $ABC$ с~центром~$I$ касается
сторон $AC$ и~$AB$ в~точках $E$ и~$F$.
Точка~$J$ на~$EF$ такова, что прямая~$BJ$ параллельна $AC$.
Пусть $CJ \cap AB = K$.
Докажите, что прямая~$IK$ параллельна $EF$.
% https://artofproblemsolving.com/community/c6h1098261p4936313

\item
Вписанная окружность треугольника $ABC$ с~центром~$I$ касается стороны~$AC$
в~точке~$Q$;
точка~$E$~--- середина стороны~$AC$, а~$K$~--- ортоцентр треугольника $BIC$.
Докажите, что прямая $KQ \perp IE$.
% https://artofproblemsolving.com/community/c6h597376p3545118

\item
Дан треугольник $ABC$ и~его описанная окружность~$\omega$.
Точки $A_{1}$, $B_{1}$ и~$C_{1}$ на~прямых $BC$, $CA$ и~$AB$ соответственно
таковы, что $B_{1}$ и~$C_{1}$ лежат на~поляре точки $A_{1}$
относительно $\omega$.
Докажите, что $C_{1}$ лежит на~поляре точки $B_{1}$ относительно $\omega$.
% https://artofproblemsolving.com/community/c6h1726731p11184245

\item
Вписанная окружность треугольника $ABC$ с~центром~$I$ касается
сторон $AC$ и~$AB$ в~точках $E$ и~$F$.
Прямая~$\ell$ проходит через вершину~$C$ и~пересекает $AB$ и~$EF$
в~точках $M$ и~$N$ соответственно.
Прямая~$ME$ пересекает $CF$ в~точке~$J$.
Докажите, что $AJ \perp IN$.
% https://artofproblemsolving.com/community/c6h1137042p5316002
% https://artofproblemsolving.com/community/c6t48f6h1082931

\item
Высоты $AA_{1}$, $BB_{1}$ и~$CC_{1}$ треугольника $ABC$ пересекаются
в~точке~$H$.
Точка~$M$~--- середина~$BC$.
Пусть $T$~--- точка пересечения прямых $B_{1}C_{1}$ и~$HM$.
Докажите, что прямая~$TA_{1}$ проходит через точку пересечения касательных
к~описанной окружности треугольника $ABC$ в~точках $B$ и~$C$.
% https://artofproblemsolving.com/community/c6t31304f6h1348933

\item
Точки $R$ и~$Q$ получаются из~точки~$P$ симметрией относительно
сторон $AB$ и~$AC$ треугольника $ABC$ соответственно.
Пусть $RQ \cap BC=T$.
Докажите, что $\angle APB = \angle APC$ тогда и~только тогда, когда
$\angle APT = 90^{\circ}$.
% https://artofproblemsolving.com/community/c6h1140462p5353009

\item
Пусть $I$ и~$H$~--- центр вписанной окружности и~ортоцентр треугольника $ABC$.
Точка~$P$ на~$BC$ такова, что $PI \perp AI$, $M$~--- середина~$AP$.
Окружность с~диаметром~$IM$ пересекает вписанную в~треугольник $ABC$ окружность
в~точках $U$ и~$V$.
Докажите, что точки $U$, $V$ и~$H$ лежат на~одной прямой.
% https://artofproblemsolving.com/community/c6h1629927p10235793

\end{problems}


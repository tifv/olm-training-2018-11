% $date: 2018-11-14
% $timetable:
%   g9r1:
%     2018-11-14:
%       1:

% $build$matter[print]: [[.], [.]]

\worksheet*{Поляра или/и~инверсия или что-то еще?}

% $authors:
% - Фёдор Львович Бахарев

\begin{problems}

\item
Окружности $\omega_{1}$ и~$\omega_{2}$ перпендикулярны.
Точки $A$ и~$B$ диаметрально противоположны в~окружности~$\omega_{1}$.
Докажите, что поляра~$A$ относительно $\omega_{2}$ проходит через $B$.

\item
На~прямой, содержащей диаметр~$AB$ окружности~$\Omega$ с~центром~$O$,
за~точкой~$B$ выбрана точка~$C$.
Прямая, проходящая через $C$, пересекает $\Omega$ в~точках $D$ и~$E$.
Отрезок~$OF$~--- диаметр окружности~$\omega$, описанной около
треугольника $DOB$.
Прямая~$CF$ пересекает повторно окружность~$\omega$ в~точке~$G$.
Докажите, что точки $O$, $A$, $E$ и~$G$ лежат на~одной окружности.

\item
Вписанная окружность треугольника $ABC$ с~центром~$I$ касается
сторон $AC$ и~$AB$ в~точках $E$ и~$F$.
Точка~$J$ на~$EF$ такова, что прямая~$BJ$ параллельна $AC$.
Пусть $CJ \cap AB = K$.
Докажите, что прямая~$IK$ параллельна $EF$.

\item
Вписанная окружность треугольника $ABC$ с~центром~$I$ касается
сторон $AC$ и~$AB$ в~точках $E$ и~$F$.
Прямая~$\ell$ проходит через вершину~$C$ и~пересекает $AB$ и~$EF$
в~точках $M$ и~$N$ соответственно.
Прямая~$ME$ пересекает $CF$ в~точке~$J$.
Докажите, что $AJ \perp IN$.

\item
Докажите, что во~вписанно-описанном четырехугольнике прямая, соединяющая центры
вписанной и~описанной окружностей проходит через точку пересечения диагоналей.

\item
Диагонали четырехугольника $ABCD$, вписанного в~окружность с~центром~$O$,
различны по~длине и~пересекаются в~точке~$E$.
Точка~$P$ внутри четырехугольника такова, что
\[
    \angle PAB + \angle PCB = \angle PBC + \angle PDC = 90^\circ
\, . \]
Докажите, что точки $O$, $P$ и~$E$ коллинеарны.

\item
Точки $X$, $Y$ и~$Z$ получены путем отражения центра вписанной окружности $I$
треугольника $ABC$ относительно сторон $BC$, $CA$ и~$AB$ соответственно.
Докажите, что $AX$, $BY$ и~$CZ$ пересекаются в~одной точке.

\item
В~неравнобедренном треугольнике $ABC$ угол~$A$ равен $60^{\circ}$,
$I$ и~$O$~--- центры вписанной и~описанной окружностей.
Докажите, что серединные перпендикуляры к~отрезкам $AI$, $OI$ и~$BC$
пересекаются в~одной точке.

\item
Вписанная окружность с~центром~$I$ касается сторон $BC$, $CA$ и~$AB$
треугольника $ABC$ в~точках $K$, $L$ и~$M$ соответственно.
Прямая, проходящая через $B$ параллельно $MK$, пересекает прямые $LM$ и~$LK$
в~точках $R$ и~$S$.
Докажите, что $\angle RIS < 90^\circ$.
% IMO1998

\end{problems}


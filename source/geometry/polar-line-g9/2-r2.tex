% $date: 2018-11-14 p2
% $date$groups:
%   g9r2: 2018-11-14 p2

\worksheet{Полярная}

% $authors:
% - Фёдор Львович Бахарев

\begin{problems}

\item
Окружности $\omega_{1}$ и~$\omega_{2}$ перпендикулярны.
Точки $A$ и~$B$ диаметрально противоположны в~окружности~$\omega_{1}$.
Докажите, что поляра~$A$ относительно $\omega_{2}$ проходит через $B$.

\item
Дан треугольник $ABC$ и~его описанная окружность~$\omega$.
Точки $A_{1}$, $B_{1}$ и~$C_{1}$ на~прямых $BC$, $CA$ и~$AB$ соответственно
таковы, что $B_{1}$ и~$C_{1}$ лежат на~поляре точки~$A_{1}$
относительно $\omega$.
Докажите, что $C_{1}$ лежит на~поляре точки $B_{1}$ относительно $\omega$.

\item
Точки $X$, $Y$ и~$Z$~--- точки пересечения противоположных сторон и~точка
пересечения диагоналей четырехугольника $ABCD$, вписанного в~окружность
с~центром~$O$.
Докажите, что $O$~--- ортоцентр треугольника $XYZ$.

\item
Высоты $BB_{1}$ и~$CC_{1}$ треугольника $ABC$ пересекаются в~точке~$H$.
Прямые $B_{1}C_{1}$ и~$BC$ пересекаются в~точке~$K$.
Докажите, что прямая~$KH$ перпендикулярна медиане~$AM$.

\item
Четырехугольник $ABCD$ описан около окружности с~центром в~точке~$O$.
Прямые $AB$ и~$CD$ пересекаются в~точке~$P$, а~прямые $AD$ и~$BC$~---
в~точке~$Q$, причем отрезки $BP$ и~$DQ$ пересекаются в~точке~$A$;
$M$~--- основание перпендикуляра, опущенного из~точки $O$ на~$PQ$.
Докажите, что
\\
\subproblem
точка пересечения диагоналей четырехугольника $ABCD$ лежит на~прямой~$OM$;
\\
\subproblem
$\angle BMO = \angle DMO$.

\item
Вписанная окружность треугольника $ABC$ с~центром~$I$ касается
сторон $AC$ и~$AB$ в~точках $E$ и~$F$.
Точка~$J$ на~$EF$ такова, что прямая~$BJ$ параллельна $AC$.
Пусть $CJ \cap AB = K$.
Докажите, что прямая $IK$ параллельна $EF$.

\item
На~прямой, содержащей диаметр~$AB$ окружности~$\Omega$ с~центром~$O$,
за~точкой~$B$ выбрана точка~$C$.
Прямая, проходящая через $C$, пересекает $\Omega$ в~точках $D$ и~$E$.
Отрезок~$OF$~--- диаметр окружности~$\omega$, описанной около
треугольника $DOB$.
Прямая~$CF$ пересекает повторно окружность~$\omega$ в~точке~$G$.
Докажите, что точки $O$, $A$, $E$ и~$G$ лежат на~одной окружности.

\item
Докажите, что во~вписанно-описанном четырехугольнике прямая, соединяющая центры
вписанной и~описанной окружностей проходит через точку пересечения диагоналей.

\end{problems}


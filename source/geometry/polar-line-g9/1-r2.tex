% $date: 2018-11-12 p3
% $date$groups:
%   g9r2: 2018-11-12 p3

\worksheet{Предполярный разнобой}

% $authors:
% - Фёдор Львович Бахарев

\begin{problems}

\item
\subproblem
Внутри окружности зафиксирована точка~$A$, а~на~окружности взяты произвольные
точки $B$ и~$C$ так, что $\angle BAC = 90^{\circ}$.
Найдите геометрическое место точек пересечения касательных к~окружности,
проведенных в~точках $B$ и~$C$.
\\
\subproblem
Внутри окружности зафиксирована точка~$A$ и~через нее проводятся всевозможные
хорды~$BC$.
Найдите геометрическое место точек пересечения касательных к~окружности,
проведенных в~точках $B$ и~$C$.
\\
\subproblem
Через точку~$A$ вне окружности проводятся всевозможые секущие, пересекающие
окружность в~точках $B$ и~$C$.
Найдите геометрическое место точек пересечения касательных к~окружности,
проведенных в~точках $B$ и~$C$.

\item
Стороны $AB$, $BC$, $CD$ и~$DA$ описанного четырехугольника $ABCD$ касаются
вписанной окружности в~точках $K$, $L$, $M$ и~$N$ соответственно.
Отрезок~$KM$ пересекает диагональ~$AC$ в~точке~$X$.
Докажите, что
\\
\subproblem
$AX / XC = AK / MC$;
\\
\subproblem
диагонали четырехугольников $ABCD$ и~$KLMN$ пересекаются в~одной точке.

\item
Через точку~$A$, лежащую вне окружности проведены две секущие.
Первая пересекает окружность в~точках $K$ и~$L$ ($L$ лежит между $A$ и~$K$),
а~вторая~--- в~точках $M$ и~$N$ ($M$ лежит между $A$ и~$N$).
Касательные в~точках $K$ и~$L$ пересекаются в~точке~$X$, касательные
в~точках $M$ и~$N$~--- в~точке~$Y$.
Докажите, что
\\
\subproblem
точка пересечения диагоналей четырехугольника $KLMN$ лежит на~прямой~$XY$;
\\
\subproblem
прямые $LM$ и~$KN$ пересекаются на~прямой~$XY$ или параллельны ей.

\item
Дана окружность~$\omega$ с~центром~$O$ и~четыре такие точки
$A$, $B$, $A'$ и~$B'$, что $A'$ и~$B'$ соответственно симметричны
точкам $A$ и~$B$ относительно $\omega$.
Через точку~$A'$ проведена прямая~$a$, перпендикулярная $OA$,
а~через точку~$B'$~--- прямая~$b$, перпендикулярная $OB$.
Докажите, что если $A$ лежит на~$b$, то~$B$ лежит на~$a$.
\\
\subproblem
Рассмотрите случай, когда $A$ и~$B$ лежат вне окружности.
\\
\subproblem
Рассмотрите все остальные случаи.

\item
Окружности $\omega_{1}$ и~$\omega_{2}$ пересекаются в~точках $A$ и~$B$.
Касательные в~точке~$A$ к~$\omega_{1}$ и~$\omega_{2}$ повторно пересекают
окружности в~точках $C$ и~$D$.
Точка~$E$ такова, что $B$~--- середина отрезка~$AE$.
Докажите, что точки $A$, $C$, $D$ и~$E$ лежат на~одной окружности.

\item
В~четырехугольнике $ABCD$ выполнено соотношение
$\angle A + \angle C = 90^{\circ}$.
Докажите, что
\[
    (AB \cdot CD)^2 + (BC \cdot AD)^2 = (AC \cdot BD)^2
\, . \]

\item
Четыре окружности $\omega_{1}$, $\omega_{2}$, $\omega_{3}$ и~$\omega_{4}$
попарно касаются друг друга внешним образом.
Обозначим через $A_{ij}$ точку касания окружностей $\omega_{i}$ и~$\omega_{j}$.
Докажите, что прямые $A_{12}A_{34}$, $A_{13}A_{24}$ и~$A_{14}A_{23}$
пересекаются в~одной точке.

\item
Точка~$I$~--- центр вписанной окружности треугольника $ABC$.
Меньшую из~окружностей, проходящих через точку~$I$ и~касающихся
сторон $AB$ и~$AC$, обозначим через $\omega_{A}$.
Аналогично определим $\omega_{B}$ и~$\omega_{C}$.
Вторые точки пересечения
$\omega_{A}$ и~$\omega_{B}$,
$\omega_{B}$ и~$\omega_{C}$,
$\omega_{C}$ и~$\omega_{A}$
обозначим через $C_{1}$, $A_{1}$, $B_{1}$ соответственно.
Докажите, что центры окружностей, описанных около
треугольников $AA_{1}I$, $BB_{1}I$ и~$CC_{1}I$, лежат на~одной прямой.

\end{problems}


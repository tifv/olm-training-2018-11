% $date: 2018-11-12 p3
% $date$groups:
%   g9r1: 2018-11-12 p3

\worksheet{Предполярный разнобой}

% $authors:
% - Фёдор Львович Бахарев

\begin{problems}

\item
\subproblem
Внутри окружности зафиксирована точка~$A$, а~на~окружности взяты произвольные
точки $B$ и~$C$ так, что $\angle BAC = 90^{\circ}$.
Найдите геометрическое место точек пересечения касательных к~окружности,
проведенных в~точках $B$ и~$C$.
\\
\subproblem
Через точку~$A$ проводятся всевозможые секущие, пересекающие окружность
в~точках $B$ и~$C$.
Найдите геометрическое место точек пересечения касательных к~окружности,
проведенных в~точках $B$ и~$C$.

\item
Стороны $AB$, $BC$, $CD$ и~$DA$ описанного четырехугольника $ABCD$ касаются
вписанной окружности в~точках $K$, $L$, $M$ и~$N$ соответственно.
Отрезок~$KM$ пересекает диагональ~$AC$ в~точке~$X$.
Докажите, что
\\
\subproblem
$AX / XC = AK / MC$;
\\
\subproblem
диагонали четырехугольников $ABCD$ и~$KLMN$ пересекаются в~одной точке.

\item
Через точку~$A$ проводятся всевозможные пары секущих к~окружности~$\omega$
с~центром~$O$.
Первая пересекает окружность в~точках $K$ и~$L$, а~вторая~---
в~точках $M$ и~$N$.
\\
\subproblem
Докажите, что всевозможные точки пересечения прямых $KM$ и~$NL$ лежат на~одной
прямой.
\\
\subproblem
Докажите, что построенная таким образом прямая проходит через инверсный образ
точки~$A$ и~перпендикулярна $AO$.

\item
Дана окружность~$\omega$ с~центром~$O$ и~четыре такие точки
$A$, $B$, $A'$ и~$B'$, что $A'$ и~$B'$ соответственно симметричны
точкам $A$ и~$B$ относительно $\omega$.
Через точку~$A'$ проведена прямая~$a$, перпендикулярная $OA$,
а~через точку~$B'$~--- прямая~$b$, перпендикулярная $OB$.
Докажите, что если $A$ лежит на~$b$, то~$B$ лежит на~$a$.

\item
Точки $X$, $Y$ и~$Z$~--- точки пересечения противоположных сторон и~точка
пересечения диагоналей четырехугольника $ABCD$, вписанного в~окружность
с~центром~$O$.
Докажите, что $O$~--- ортоцентр треугольника $XYZ$.

\item
Окружности $\omega_{1}$ и~$\omega_{2}$ пересекаются в~точках $A$ и~$B$.
Касательные в~точке~$A$ к~$\omega_{1}$ и~$\omega_{2}$ повторно пересекают
окружности в~точках $C$ и~$D$.
Точка~$E$ такова, что $B$~--- середина отрезка~$AE$.
Докажите, что точки $A$, $C$, $D$ и~$E$ лежат на~одной окружности.

\item
Четыре окружности $\omega_{1}$, $\omega_{2}$, $\omega_{3}$ и~$\omega_{4}$
попарно касаются друг друга внешним образом.
Обозначим через $A_{ij}$ точку касания окружностей $\omega_{i}$ и~$\omega_{j}$.
Докажите, что прямые $A_{12}A_{34}$, $A_{13}A_{24}$ и~$A_{14}A_{23}$
пересекаются в~одной точке.

\item
Четырехугольник $ABCD$ описан около окружности с~центром в~точке~$O$.
Прямые $AB$ и~$CD$ пересекаются в~точке~$P$, а~прямые $AD$ и~$BC$~---
в~точке~$Q$, причем отрезки $BP$ и~$DQ$ пересекаются в~точке~$A$;
$M$~--- основание перпендикуляра, опущенного из~точки~$O$ на~$PQ$.
Докажите, что
\\
\subproblem
точка пересечения диагоналей четырехугольника $ABCD$ лежит на~прямой~$OM$;
\\
\subproblem
$\angle BMO = \angle DMO$.

\item
Вписанная окружность треугольника $ABC$ с~центром~$I$ касается стороны~$AC$
в~точке~$Q$;
точка~$E$~--- середина стороны~$AC$, а~$K$~--- ортоцентр треугольника $BIC$.
Докажите, что $KQ \perp IE$.

\end{problems}


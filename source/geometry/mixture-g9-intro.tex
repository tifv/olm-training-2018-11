% $date: 2018-11-06
% $timetable:
%   g9r1:
%     2018-11-06:
%       3:
%   g9r2:
%     2018-11-06:
%       1:

\worksheet*{Вступительный разнобой}

% $authors:
% - Фёдор Львович Бахарев

\begin{problems}

\item
Дана точка~$A$ и~окружность $\omega$, не~проходящая через $A$.
Докажите, что
\\
\subproblem
все окружности, проходящие через точку~$A$ и~пересекающие окружность~$\omega$
в~диаметрально противоположных точках, проходят одновременно через некоторую
точку~$A'$, отличную от~$A$;
\\
\subproblem
все окружности, проходящие через точку~$A$ и~перпендикулярные
к~окружности~$\omega$, проходят одновременно через некоторую точку~$A'$,
отличную от~$A$.
(Окружности называются перпендикулярными, если касательные, проведенные к~ним
в~точке пересечения, перпендикулярны.)

\item
В~параллелограмме $ABCD$ дана точка~$M$, такая что $\angle MAD = \angle MCD$.
Докажите, что $\angle MBA = \angle MDA$.

\item
Внутри окружности зафиксирована точка~$A$, а~на~окружности взяты произвольные
точки $B$ и~$C$ так, что $\angle BAC = 90^{\circ}$.
Найдите геометрическое место середин хорд~$BC$.

\item
На~сторонах $AB$, $BC$ и~$CA$ треугольника $ABC$ взяты точки $P$, $Q$ и~$R$
соответственно.
Докажите, что центры описанных окружностей треугольников $APR$, $BPQ$ и~$CQR$
образуют треугольник, подобный треугольнику $ABC$.

\item
Медиана~$AM$ треугольника $ABC$ пересекает вписанную в~него окружность
в~точках $X$ и~$Y$.
Известно, что $AB = AC + AM$.
Найдите $\angle XIY$, где $I$~--- центр вписанной окружности.

\item
В~остроугольном треугольнике $ABC$ точки $O$ и~$I$~--- центры описанной
и~вписанной окружностей соответственно.
При этом $\angle OIA = 30^\circ$.
Докажите, что один из~углов $B$ и~$C$ равен $60^\circ$.

\item
На~отрезке~$MN$ построены подобные, одинаково ориентированные треугольники
$AMN$, $NBM$ и~$MNC$.
Докажите, что треугольник $ABC$ подобен всем этим треугольникам, а~центр его
описанной окружности равноудален от~точек $M$ и~$N$.

\item
Окружность, проходящая через вершины $A$ и~$B$ треугольника $ABC$, пересекает
сторону~$BC$ в~точке~$D$.
Окружность, проходящая через вершины $B$ и~$C$, пересекает сторону~$AB$
в~точке~$E$ и~первую окружность вторично в~точке~$F$.
Оказалось, что точки $A$, $E$, $D$, $C$ лежат на~окружности с~центром~$O$.
Докажите, что угол $BFO$~--- прямой.

\end{problems}


% $date: 2018-11-12
% $timetable:
%   g10r2:
%     2018-11-12:
%       2:

\worksheet*{Гармоничное}

% $authors:
% - Фёдор Львович Бахарев

\begin{claim}{Определение}
Вписанный четырехугольник $ABCD$ называется гармоническим, если
$AB \cdot CD = AD \cdot BC$.
\end{claim}

\begin{problems}

\item
Четырехугольник $ABCD$ вписан в~окружность.
Докажите, что касательные к~окружности в~точках $B$ и~$D$ пересекаются
на~прямой~$AC$ тогда и~только тогда, когда $ABCD$ гармонический.

\item
Четырехугольник $ABCD$ вписан в~окружность, $M$~--- середина диагонали~$AC$.
Докажите, что $\angle AMB = \angle ADC$ тогда и~только тогда, когда
$ABCD$ гармонический.

\item
Четырехугольник $ABCD$ вписан в~окружность, $M$~--- середина диагонали~$AC$.
Докажите, что $\angle AMB = \angle DCB$ тогда и~только тогда, когда $ABCD$
гармонический.

\item
Докажите, что вписанный четырехугольник является гармоническим тогда и~только
тогда, когда диагональ~$AC$ является симедианой треугольника $ABD$.

\item
Через точку~$A$ проводятся всевозможные секущие, пересекающие окружность
в~точках $B$ и~$C$.
Найдите геометрическое место точек пересечения касательных к~окружности,
проведенных в~точках $B$ и~$C$.

\item
Пусть $ABCD$~--- гармонический четырехугольник, $P$~--- некоторая точка.
Прямые $PA$, $PB$, $PC$ и~$PD$ повторно пересекают окружность, описанную
около четырехугольника $ABCD$, в~точках $A'$, $B'$, $C'$ и~$D'$ соответственно.
Докажите, что четырехугольник $A'B'C'D'$ гармонический.

\item
Дан гармонический четырехугольник $ABCD$ и~прямая~$\ell$, параллельная
касательной, проведенной в~точке~$A$ к~его описанной окружности.
Прямые $AB$, $AC$ и~$AD$ пересекают прямую~$\ell$ в~точках $B'$, $C'$ и~$D'$
соответственно.
Докажите, что $C'$~--- середина отрезка $B'D'$.

\item
В~окружности~$S$ проведены две параллельные хорды $AB$ и~$CD$.
Прямая, проведенная через $C$ и~середину~$AB$, вторично пересекает окружность
в~точке~$E$.
Точка~$K$~--- середина отрезка~$DE$.
Докажите, что $\angle AKE = \angle BKE$.

\item
Дана окружность с~центром~$O$ и~диаметром~$BC$.
Из~точки~$A$ на~окружности опускается перпендикуляр~$AH$ на~$BC$ и~берется его
середина~$M$.
Прямая~$BM$ повторно пересекает окружность в~точке~$N$.
Касательная в~точке~$N$ к~окружности пересекает $AC$ в~точке~$P$.
Найдите геометрическое место точек~$P$, когда точка~$A$ двигается
по~окружности.

\item
Окружность, вписанная в~треугольник $ABC$, касается стороны~$AC$ в~точке~$D$.
Отрезок~$BD$ повторно пересекает окружность в~точке~$E$.
Точки $F$ и~$G$ на~окружности таковы, что $FE \parallel BC$
и~$GE \parallel BA$.
Докажите, что прямая, соединяющая центры вписанных окружностей треугольников
$DEF$ и~$DEG$, перпендикулярна биссектрисе угла~$B$.

\item
Точка~$O$~--- центр описанной окружности~$\omega$ треугольника $ABC$, а~$M$~---
середина стороны~$BC$.
Точки $E$ на~$AC$ и~$F$ на~$AB$ таковы, что $ME = MF$.
Окружность, описанная около треугольника $AEF$, пересекает повторно $\omega$
в~точке~$P$, отрезок $AU$~--- ее диаметр.
Докажите, что четырехугольник $PEUF$ гармонический.

\end{problems}


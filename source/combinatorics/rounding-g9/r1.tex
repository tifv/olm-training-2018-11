% $date: 2018-11-12 p2
% $date$groups:
%   g9r1: 2018-11-12 p2

\worksheet{Целые и нецелые}

% $authors:
% - Александр Васильевич Шаполовалов

% $content[-ashap-guard,contained]:
% - newpage: false
%   content:
%   - .[ashap-guard]
%   - /ashap-link

\begin{exercises}

\itemy{0}
У~аптекаря есть три гирьки, с~помощью которых он
одному покупателю отвесил 100\,\text{г} йода,
другому~--- 101\,\text{г} мёда,
а~третьему~--- 102\,\text{г} перекиси водорода.
(Гирьки он ставил только на~одну чашку весов.)
Могла~ли каждая гирька быть легче 90\,\text{г}?

\item
Есть 10 карточек, на~каждой по~числу.
Для каждого набора карточек вычислили его сумму.
Не~все суммы~--- целые.
Какое наибольшее количество сумм может быть целыми?

\end{exercises}

\begin{claim}{Определение}
Назовем \emph{округлением} замену нецелого числа на~одно из~двух ближайших
целых (с~недостатком или с~избытком), а~целое пусть при округлении не~меняется.
Например, $3{,}14$ можно округлить до~$3$ или до~$4$.
\end{claim}

\begin{exercises}

\item
\subproblem
Записано верное равенство: 100 слагаемых и~их сумма.
Докажите, что все нецелые числа можно округлить до~целого так, чтобы равенство
осталось верным.
\\
\subproblem
В~вершинах куба выписаны числа, а~на~каждом ребре~--- сумма чисел в~его концах.
Докажите, что можно все 20~чисел округлить так, чтобы по-прежнему на~каждом
ребре стояла сумма чисел в~его концах.
\\
\subproblem
В~вершинах пятиугольной призмы выписаны числа, а~на~каждом ребре~--- сумма
чисел в~его концах.
Всегда~ли можно все 25 чисел округлить так, чтобы по-прежнему на~каждом ребре
стояла сумма чисел в~его концах?

\item
В~10~кошельках лежали монеты так, что веса любых двух монет из~одного кошелька
отличались не~более чем на~1\,\text{г} (веса монет могут быть нецелыми).
Монеты смешали и~положили в~непрозрачный мешок.
Саша про веса монет заранее ничего не~знает.
Он вынимает одну монету из~мешка, взвешивает, затем кладет монету в~одну
из~имеющихся 20~коробок, вынимает следующую монету и~т.д.
(Положив монету в~коробку, потом её уже нельзя переложить).
Докажите, что Саша может действовать так, чтобы в~каждой коробке веса любых
двух монет отличались не~более чем на~1\,\text{г}.

\item
Даны $9$ чисел $a_{1}$, $a_{2}$, \ldots, $a_{9}$.
Известно, что не~все числа $2 a_{1}$, $2 a_{2}$, \ldots, $2 a_{9}$~--- целые.
Какое наибольшее число целых может быть среди попарных сумм $a_{i} + a_{j}$
($i \neq j$)?

\item
Надо сделать набор из~пяти гирь, с~помощью которых можно уравновесить любой
целый вес от~$5\,\text{г}$ до~$10\,\text{г}$
(гири кладутся на~одну чашку весов, измеряемый вес~--- на~другую, веса гирь
не~обязательно целые).
Одна гиря делается из~золота, каждая из~остальных не~тяжелее золотой.
Каким наименьшим количеством золота можно обойтись?

\item
Шесть команд в~однокруговом турнире набрали 10, 7, 6, 6, 3 и~3 очка.
Сколько очков начислялось за~победу, если за~ничью давали 1 очко,
а~за~поражение 0?

\item
\subproblem
В~клетки прямоугольной таблицы вписаны числа.
Выписаны также суммы для каждой строки, для каждого столбца и~для всей таблицы.
Все эти суммы оказались целыми.
Докажите, что все числа в~таблице можно округлить так, чтобы все суммы
по-прежнему сходились.
\\
\subproblem
То~же, но~суммы могут быть не~целыми.
Докажите, что можно округлить числа в~таблице и~суммы так, чтобы всё сходилось.

\end{exercises}

\subsubsection*{Зачетные задачи}

\begin{problems}

\item
Вес каждой гирьки набора~--- нецелое число грамм.
Ими можно уравновесить любой целый вес от~1\,\text{г} до~40\,\text{г}
(гири кладутся на~одну чашку весов, измеряемый вес~--- на~другую).
Каково наименьшее число гирь в~таком наборе?

\item
Алёна и~Боря независимо друг от~друга складывают одни и~те~же $n$ чисел в~одном
порядке.
На~каждом шаге (написав первое число, прибавив второе и~т.\,д.) они делают
следующее.
Если дробная часть полученной суммы меньше $0{,}5$, то~Алёна округляет
до~ближайшего меньшего целого, а~Боря не~округляет.
Если~же дробная часть больше или равна $0{,}5$, то~Боря округляет до~ближайшего
большего целого, а~Алёна не~округляет.
Какова наибольшая возможная разность между результатами Бори и~Алёны?

\itemy{5}
Имеется набор из~20 гирь, с~помощью которых можно взвесить любой целый вес
от~1\,\text{г} до~1997\,\text{г}
(гири кладутся на~одну чашку весов, измеряемый вес~--- на~другую).
Каков минимально возможный вес самой тяжелой гири такого набора?

\itemy{6}
\subproblem
Из~колоды отложили часть карт.
Докажите, что оставшиеся можно разделить между двумя игроками так, чтобы у~них
общее число карт, число карт каждой масти и~число карт каждого достоинства
отличалось не~более чем на~1.
\\
\subproblem
То~же, но~карты делятся между тремя игроками.

\end{problems}

% www.ashap.info/Uroki/Mosbory/2018o


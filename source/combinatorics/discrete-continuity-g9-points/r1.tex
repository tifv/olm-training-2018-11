% $date: 2018-11-15 p3
% $date$groups:
%   g9r1: 2018-11-15 p3

\worksheet{Непрерывная комбинаторика: точки на отрезке и окружности}

% $authors:
% - Александр Васильевич Шаполовалов

% $content[-ashap-guard,contained]:
% - newpage: false
%   content:
%   - .[ashap-guard]
%   - /ashap-link

В~некоторых задачах возникают комбинации из~конечного числа объектов, но~сами
объекты выбираются из~бесконечного набора, заданного непрерывным параметром или
параметрами.
Хорошей моделью в~таких задачах служат наборы точек на~прямой и~окружности;
работает обобщенный принцип Дирихле.

\begin{exercises}

\item
Хозяйка испекла для гостей пирог.
\\
\subproblem
За~столом может оказаться либо $p$ человек, либо $q$.
Как заранее разрезать пирог не~более чем на~$p + q - 1$ кусков
(не~обязательно равных), чтобы в~любом случае его можно было раздать поровну?
\\
\subproblem
За~столом может оказаться либо $4$ человека, либо $6$.
На~какое минимальное количество кусков нужно заранее разрезать пирог, чтобы
в~любом случае его можно было раздать поровну?
\\
\subproblem
А~если либо 9, либо 15 гостей?

\item
Есть 11 гирь, каждая весит меньше 100\,\text{г}.
\\
\subproblem
Выписаны по~разу веса всевозможных пар из~этих гирь.
Докажите, что какие-то два веса отличаются менее чем на~4\,\text{г}.
\\
\subproblem
Известно, что каждая гиря отличается по~весу более чем на~4\,\text{г}
от~любой другой гири.
Докажите, что можно выбрать 4 гири и~разложить их на~две пары так, чтобы веса
пар отличались меньше, чем на~4\,\text{г}.

\item
\subproblem
\emph{Лемма о~кратной подсумме.}
В~ряд выписали $n$ целых чисел.
Докажите, что в~это ряду можно подчеркнуть одно или несколько подряд идущих
чисел так, чтобы их сумма делилась на~$n$.
\\
\subproblem
В~ряд выписаны действительные числа
$a_{1}$, $a_{2}$, $a_{3}$, \ldots, $a_{1000}$.
Докажите, что можно выделить одно или несколько стоящих рядом чисел так, что их
сумма будет отличаться от~целого числа меньше, чем на~$0{,}001$.

\end{exercises}

С~группами точек можно поступать как с~одним целым: переворачивать, сдвигать,
сжимать или растягивать.
Удачно выбранная операция помогает решить задачу.

\begin{exercises}

\item
\subproblem
Известно, что число $a$ положительно, а~неравенство $1 < x a < 2$
имеет ровно $3$~решения в~целых числах.
Сколько решений в~целых числах может иметь неравенство $2 < x a < 3$?
\\
\subproblem
Известно, что число $a$ положительно, а~неравенство $10 < a^{x} < 100$
имеет ровно $5$~решений в~натуральных числах.
Сколько таких решений может иметь неравенство $100 < a^{x} < 1000$?

\end{exercises}

Полезно помнить, что на~любом, сколь угодно малом, интервале найдется
рациональное число.

\begin{exercises}

\item
Даны 100 различных чисел.
Докажите, что можно умножить все числа на~одно и~то~же рациональное число так,
чтобы ровно половина из~них стала больше 1000.

\item
\subproblem
Даны $3$ пары положительных чисел:
$a_{1} < b_{1}$, $a_{2} < b_{2}$, $a_{3} < b_{3}$.
Числа в~каждой паре разрешается увеличить в~целое число раз, для каждой пары
в~своё.
Докажите, что можно добиться, чтобы все $a_{i}$ стали меньше всех $b_{j}$.
\\
\subproblem
Есть сто картинок, на~каждой~--- взрослый и~ребенок ростом поменьше
(все двести человек на~картинках разные).
Из~них надо собрать одну большую картину.
При этом разрешается уменьшать видимый рост людей в~целое число раз: эти целые
числа могут быть разными для людей с~разных картинок и~должны быть одинаковыми
для людей с~одной картинки.
Докажите, что можно добиться, чтобы на~большой картине каждый взрослый имел
рост больше каждого ребенка.

\end{exercises}

Среди всевозможных отрезков короче данного нет самого длинного: для каждого
найдется еще длиннее.
Для любой точки на~интервале есть точка, более близкая к~его концу.

\begin{exercises}

\item\jeolmlabel{/combinatorics/discrete-continuity-g9-points/r1/:exercise:7}
Двое по~очереди отмечают точки на~окружности: первый~--- красным цветом,
второй~--- синим.
Когда отмечено по~2 точки каждого цвета, игра заканчивается.
Затем каждый игрок находит на~окружности дугу наибольшей длины с~концами своего
цвета, на~которой больше нет отмеченных точек.
У~кого длина дуги больше~--- тот выиграл (в~случае равенства длин дуг, а~также
при отсутствии таких дуг у~обоих игроков~--- ничья).
Кто из~играющих может всегда выигрывать, как~бы ни~играл соперник?

\end{exercises}

\subsubsection*{Зачетные задачи}

\begin{problems}

\item
У~Пети было 20 камней нецелого веса, выложенных по~кругу.
Для каждого камня Петя сделал взвешивание, положив этот камень на~одну чашу
весов, а~пару его соседей~--- на~другую чашу, и~записал результат:
тяжелее этот камень или легче (равенства ни~разу не~случилось).
Докажите, что Вася может подменить все камни на~камни целого веса так, чтобы
такая~же проверка дала точно те~же результаты.

\item
\subproblem
Есть 10 яблок весом от~50\,\text{г} до~100\,\text{г}.
Докажите, что из~них можно можно выбрать два непересекающихся набора, чьи веса
отличаются менее чем на~1\,\text{г}.
\\
\subproblem
То~же, но~в~наборах должно быть одинаковое число яблок.

\item
Купившему головку сыра весом 3\,\text{кг} магазин предлагает призовую игру.
Покупатель режет головку на~4 куска, а~продавец выбирает из~этих кусков один
или несколько и~раскладывает их на~одну или на~обе чаши весов.
При неравновесии продавец обязан за~счет магазина добавить призовой кусок сыра,
уравновешивающий чаши.
Продавец старается сделать приз поменьше, а~покупатель~--- побольше.
Найдите вес призового куска при наилучших действиях сторон.

\item
Задача~\jeolmref{/combinatorics/discrete-continuity-g9-points/r1/:exercise:7}
когда каждый отмечает по~$N > 2$ точек.

\end{problems}

% www.ashap.info/Uroki/Mosbory/2018o


% $date: 2018-11-05
% $date$groups:
%   g9r1: 2018-11-05 p3
%   g9r2: 2018-11-05 p2

% Серия 1. Асимптотика
\worksheet{Асимптотика}

% $authors:
% - Григорий Алексеевич Юргин

\begin{problems}

\item
На~бесконечной клетчатой доске двое по~очереди делают ходы.
Первый ставит в~пустую клетку один крестик, потом второй~--- сто ноликов.
Первый выигрывает, если нашелся прямоугольник с~вершинами в~крестиках, стороны
которого параллельны линиям сетки.
Может~ли первый обеспечить себе победу?

\item
Докажите, что плоскость нельзя покрыть 99 параболами с~их внутренностями.

\item
На~клетчатой плоскости проведены прямые
$\ell_{1}$, $\ell_{2}$, \ldots, $\ell_{99}$,
каждая из~которых проходит через пару узлов целочисленной решётки.
Для каждого конечного множества~$M$, состоящего из~узлов целочисленной решётки,
определим его \emph{тень} как набор $( M_{1}, M_{2}, \ldots, M_{99} )$ проекций
множества на~прямые $\ell_{1}$, $\ell_{2}$, \ldots, $\ell_{99}$ соответственно.
Докажите, что существует миллион попарно несовпадающих конечных множеств,
состоящих из~узлов целочисленной решётки, тени которых совпадают.

%\item
%Вася провёл 100 прямых, каждая из~которых проходит через пару узлов
%целочисленной решётки.
%Докажите, что Петя может так выбрать 100 различных фигур, каждая из~которых
%состоит из~конечного числа целочисленных узлов, так, чтобы для каждой васиной
%прямой проекции всех петиных фигур на~эту прямую были одинаковы.
% Изначальная формулировка. Она совсем непонятная.

\item
Существует~ли отображение из~некоторого трёхмерного шара в~некоторый плоский
круг, для каждой пары точек не~уменьшающее расстояние между ними?

\item
Плоскость разбита на~равные многоугольники, причём в~каждом многоугольнике
содержится ровно по~одной целой точке, и~на~границе целых точек нет.
Докажите, что площадь многоугольников равна единице.

\item
Докажите, что для любых натуральных $n$ и~$k$ существует такой полный
ориентированный граф~$G$, обладающий свойством: для любого множества~$A$,
состоящего из~$k$~вершин графа~$G$, существует такое множество~$B$, состоящее
из~$n$~вершин графа~$G$, что $A$ и~$B$ не~имеют общих вершин и~нет никаких
стрелок из~$A$ в~$B$.

\item
Докажите, что существует натуральное число~$n$, для которого уравнение
\[ x^3 + y^3 + z^3 + t^3 = n \]
имеет не~меньше миллиона решений в~натуральных числах.

\item
Из~клетчатой плоскости выбросили все клетки, обе координаты которых делятся
на~100.
Можно~ли все оставшиеся клетки обойти шахматным конём, побывав на~каждой ровно
по~одному разу?

\item
Дано множество~$V$ из~$n$~вершин, пронумерованных числами от~$1$ до~$n$.
Граф~$G$ на~вершинах~$V$ называется \emph{графом хорд,} если существует такое
расположение $n$ пронумерованных хорд некоторой окружности, что вершины
$i$ и~$j$ смежны в~$G$ тогда и~только тогда, когда хорды $i$ и~$j$
пересекаются.
Верно~ли, что для любого $n$ любой граф с~множеством вершин $V$ можно
представить как объединение не~более чем $10$ графов хорд?

\end{problems}


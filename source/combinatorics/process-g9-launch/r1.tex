% $date: 2018-11-14 p3
% $date$groups:
%   g9r1: 2018-11-14 p3

\worksheet{Запусти процесс}

% $authors:
% - Александр Васильевич Шаполовалов

% $content[-ashap-guard,contained]:
% - newpage: false
%   content:
%   - .[ashap-guard]
%   - /ashap-link

Запустив процесс, можно <<по~цепочке>> распространить свойство.

\begin{exercises}

\item
В~последовательности чисел каждый член (кроме первого) на~18 больше суммы
двух своих соседей.
20-й член равен 20,18.
Докажите, что в~последовательности никакие два соседних члена не~являются оба
целыми числами.

\end{exercises}

Получить искомую конструкцию можно методом последовательных улучшений.
Например, увеличивая на~каждом шаге число деталей, поставленных на~место.

\begin{exercises}

\item
\subproblem
Есть несколько кусков сыра, каждый~--- не~тяжелее 100\,\text{г}.
Докажите, что их все можно разложить на~две кучки так, чтобы веса кучек
отличались не~более чем на~100\,\text{г}.
\\
\subproblem
Есть 20 камней неизвестного веса и~двухчашечные весы без гирь.
Докажите, что сделав не~более 19 взвешиваний, можно все камни можно разложить
на~две кучки так, чтобы веса кучек отличались не~более чем на~вес самого
тяжелого камня.

\end{exercises}

Запускаясь, надо понимать, какие промежуточные ситуации получаются в~процессе,
и~как делать очередной шаг в~разных ситуациях.

\begin{exercises}

\item
Ученики школы посещают кружки.
Докажите, что можно несколько школьников принять в~пионеры так, чтобы в~каждом
кружке был хотя~бы один пионер и~для любого пионера нашелся кружок, в~котором
он был~бы единственным пионером.

\item
Среди 50 школьников каждый знаком не~менее чем с~25 другими.
Докажите, что можно их разбить на~группы из~2 или 3 человек так, чтобы каждый
был знаком со~всеми в~своей группе.

\end{exercises}

При сборке детали не~обязательно добавлять по~одной.
Можно соединять и~куски из~нескольких деталей, уменьшая общее число кусков.

\begin{exercises}

\item
На~кольцевой дороге стоят несколько одинаковых автомобилей.
Известно, что общего количества бензина в~них достаточно на~полный круг
по~кольцу.
Докажите, что найдется автомобиль, который сможет проехать полный круг, забирая
бензин у~других автомобилей по~мере проезда мимо них.

\end{exercises}

Делая очередной шаг, думайте не~только о~конечной цели, но~и~о~возможности для
следующего шага.

\begin{exercises}

\item
Есть 100 конфет 5 сортов, каждого сорта не~более 50 штук.
Докажите, что конфеты можно разбить на~50 пар так, чтобы в~каждой паре конфеты
были разного сорта.

\item
На~окружности отмечено 300 точек: по~100 точек синего, красного и~зелёного
цветов.
Докажите, что можно провести 150 отрезков с~концами в~этих точках, соблюдая
такие правила:
(1)~никакие два отрезка не~пересекаются (даже в~концах);
(2)~концы каждого отрезка~--- разного цвета.

\end{exercises}

Свяжите с~конструкцией величину, меняющийся в~одну сторону при улучшениях
\emph{(полуинвариант).}
Если полуинваринт нельзя менять бесконечно, то~его крайнее значение даст
искомый результат \emph{(принцип крайнего),} или докажет, что исходная
конструкция невозможна \emph{(бесконечный спуск).}

\begin{exercises}

\item
На~олимпиаде у~каждого участника не~более 25 знакомых.
Докажите, что можно рассадить всех участников по~трём аудиториям так, чтобы
у~каждого в~его аудитории было не~более 8 знакомых.

\item
На~координатной плоскости лежит правильный пятиугольник.
Докажите, что хотя~бы у~одной из~его вершин есть не~целая координата.

\end{exercises}

\subsubsection*{Зачетные задачи}

\begin{problems}

\item
На~клетках доски $10 \times 10$ лежит по~алмазу так, что на~соседних по~стороне
клетках веса различны.
Докажите, что алмазы можно переложить на~клетки доски $2 \times 50$ так, чтобы
по-прежнему веса на~соседних клетках были различны.

\item
На~окружности расставлено несколько положительных чисел, каждое из~которых
не~больше~1.
Докажите, что можно разделить окружность на~три дуги так, что суммы чисел
на~соседних дугах будут отличаться не~больше чем на~1.
(Если на~дуге нет чисел, то~сумма на~ней считается равной нулю.)

\setproblem{3}

\item
Шайка разбойников отобрала у~купца мешок монет.
Каждая монета стоит целое число грошей.
Оказалось, что какую~бы монету ни~отложить, оставшуюся сумму можно разделить
между разбойниками поровну.
Докажите, что если отложить одну монету, то~число монет разделится на~число
разбойников.

\item
У~Карлсона есть 100 банок с~вареньем.
Банки не~обязательно одинаковые, но~в~каждой не~больше, чем третья часть всего
варенья.
На~завтрак Карлсон может съесть поровну варенья из~любых трёх банок.
Докажите, что Карлсон может действовать так, чтобы за~некоторое количество
завтраков съесть все варенье.

\end{problems}

% www.ashap.info/Uroki/Mosbory/2018o


% $date: 2018-11-09
% $date$groups:
%   g9r1: 2018-11-09 p3
%   g9r2: 2018-11-09 p1

% Серия 4. Игры
\worksheet{Игры}

% $authors:
% - Григорий Алексеевич Юргин

\begin{problems}

\item
Изначально в~ряд выставлены $100$ мешков с~деньгами;
на~каждом мешке написано, сколько в~нём денег.
Два игрока ходят поочерёдно.
За~один ход игрок забирает себе один из~мешков с~краю текущего ряда.
Верно~ли, что при любом распределении денег по~мешкам первый игрок может играть
так, чтобы присвоить себе не~менее половины денег?

%\item
%В~концах полоски $1 \times 101$ сидят два кузнечика, каждый из~которых умеет
%прыгать на~$1$, $2$, $3$ или $4$ клетки в~направлении противоположного конца.
%Кузнечики прыгают по~очереди, каждый из~них стремится попасть в~противоположный
%конец полоски раньше соперника.
%Нельзя прыгать в~клетку, где уже сидит кузнечик.
%Какой кузнечик выиграет при правильной игре: начинающий или его соперник?

\item
Игра для двух участников состоит из~прямоугольного поля $1 \times 25$ и~$25$
фишек.
В~начальной позиции все клетки свободны.
За~один ход игрок либо выставляет новую фишку в~одну из~свободных клеток, либо
передвигает ранее выставленную фишку в~ближайшую справа свободную клетку.
Игра заканчивается, когда все клетки будут заняты фишками, причём победителем
считается игрок, сделавшим последний ход.
Игроки ходят поочередно.
Кто победит при правильной игре: начинающий или его соперник?

\item
Дана доска $2018 \times 2019$.
Два игрока ходят по~очереди.
Ход состоит в~том, чтобы закрасить некоторую связную фигурку из~$9$~клеток.
Запрещено закрашивать клетки повторно.
Проигрывает тот, кто не~может сделать ход.
Кто выиграет при правильной игре: начинающий или его соперник?
(Фигура из~клеток называется \emph{связной,} если из~любой её клетки можно
добраться до~любой другой, не~покидая фигуры и~перемещаясь между соседними
по~стороне клетками).

\item
Есть набор из~$2001$ карточки с~написанными на~них натуральными числами
от~$1$ до~$2001$.
В~начале игры у~первого игрока все карточки с~нечётными числами,
у~второго~--- с~чётными.
Игроки ходят поочерёдно, начинает первый.
За~раунд игрок, делающий ход, выкладывает одну из~своих карточек на~стол;
его оппонент после этого выкладывает одну из~своих карточек.
Тот, у~кого число больше, получает очко.
Игра заканчивается после $1000$ раундов.
Какое максимальное количество очков может набрать каждый из~игроков, вне
зависимости от~игры своего соперника?

\end{problems}


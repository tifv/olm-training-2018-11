% $date: 2018-11-16 p3
% $date$groups:
%   g9r1: 2018-11-16 p3

\worksheet{Непрерывная комбинаторика: порядок и оценки}

% $authors:
% - Александр Васильевич Шаполовалов

% $content[-ashap-guard,contained]:
% - newpage: false
%   content:
%   - .[ashap-guard]
%   - /ashap-link

В~некоторых задачах возникают комбинации из~конечного числа объектов нецелого
веса.
Важным приемом является упорядочение объектов.

\begin{exercises}

\item
Есть несколько камней, выложенных в~порядке возрастания весов.
За~какое наименьшее число взвешиваний на~чашечных весах без гирь можно
проверить или опровергнуть утверждение: Любые два камня вместе тяжелее одного?

\item
Есть $N$ упорядоченных по~весу рубинов, и~$N$ упорядоченных по~весу алмазов.
Известно, что веса всех $2N$ камней различны.
За~одну проверку можно разбить все камни на~$N$ пар (рубин--алмаз) и~для каждой
пары узнать (на~чашечных весах без гирь), какой из~двух камней в~ней тяжелее.
Требуется найти разбиение на~такие пары, в~котором ровно в~половине пар рубин
тяжелее алмаза, или доказать, что такое разбиение невозможно.
Каким наименьшим числом проверок можно обойтись?
Решите задачу для
\\
\subproblem $N = 2$;
\qquad
\subproblem $N = 4$;
\qquad
\subproblem $N = 10$.

\item
\subproblem
Имеются 300 яблок, любые два из~которых различаются по~весу не~более чем
в~два раза.
Докажите, что их можно разложить в~пакеты по~два яблока так, чтобы любые два
пакета различались по~весу не~более чем в~полтора раза.
\\
\subproblem
Имеются 300 яблок, любые два из~которых различаются по~весу не~более чем в~три
раза.
Докажите, что их можно разложить в~пакеты по~четыре яблока так, чтобы любые два
пакета различались по~весу не~более чем в~полтора раза.

\item
Есть 1000 яблок, которые надо разложить в~10 пакетов по~100 яблок в~каждом.
Оказалось, что при любой такой раскладке найдутся хотя~бы два пакета
одинакового веса.
Докажите, что
\\
\subproblem есть по~крайней мере 200 яблок одинакового веса;
\\
\subproblem есть раскладка, когда по~крайней мере 5 пакетов весят одинаково.

\end{exercises}

\subsubsection*{Разбиения с небольшой разницей}

\begin{exercises}

\item\emph{Лемма о цене игры.}
На~столе лежат несколько кусков шоколада, самый большой весит $b$.
Петя начинает, и~они с~Васей по~очереди съедают по~куску, пока не~съедят всё.
Докажите, что при наилучших действиях Васи Петя сможет съесть
\\
\subproblem не~меньше Васи;
\\
\subproblem не~более чем на~$b$ больше Васи.

\item\emph{Лемма о способах выбора.}
Есть $2n$ кусков сыра.
Докажите, что можно не~менее чем $2n$ способами разложить их по~$n$ штук на~две
чаши весов так, чтобы разность весов чаш была не~больше веса самого тяжелого
куска.

\item
В~31 ящике лежит смесь апельсинов и~бананов.
Докажите, что можно так выбрать
\\
\subproblem
16 ящиков, что в~них окажется по~весу не~менее половины всех апельсинов
и~не~менее половины всех бананов;
\\
\subproblem
11 ящиков, что в~них окажется по~весу не~менее трети всех апельсинов
и~не~менее трети всех бананов.

\end{exercises}

\subsubsection*{Зачетные задачи}

\begin{problems}

\item
Есть $20$ фруктов.
Назовем натуральное число $k < 20$ \emph{хорошим,} если найдется $k$ фруктов,
чей вес равен ровно половине общего веса.
Каково наибольшее возможное количество хороших чисел?

\item
Есть 1000 яблок, которые надо разложить в~10 пакетов по~100 яблок в~каждом.
Оказалось, что при любой такой раскладке найдутся хотя~бы два пакета
одинакового веса.
Докажите, что найдутся по~крайней мере 900 яблок одинакового веса.

\item
Есть 100 слив с~косточками.
Веса любых двух слив отличаются не~более чем вдвое, и~веса любых двух косточек тоже отличаются не~более чем вдвое.
Общий
вес косточек втрое меньше общего веса слив.
Докажите, что сливы можно разделить
на~две кучки по~50 штук так, чтобы в~каждой кучке доля косточек была  меньше
40\%.

\end{problems}

% www.ashap.info/Uroki/Mosbory/2018o


% $date: 2018-11-16 p1
% $date$groups:
%   g9r2: 2018-11-16 p1

\worksheet{Непрерывная комбинаторика: порядок и оценки}

% $authors:
% - Александр Васильевич Шаполовалов

% $content[-ashap-guard,contained]:
% - newpage: false
%   content:
%   - .[ashap-guard]
%   - /ashap-link

В~некоторых задачах возникают комбинации из~конечного числа объектов нецелого
веса.
Важным приемом является упорядочение объектов.

\begin{exercises}

\item
Есть несколько камней, выложенных в~порядке возрастания весов.
За~какое наименьшее число взвешиваний на~чашечных весах без гирь можно
проверить или опровергнуть утверждение: Любые два камня вместе тяжелее одного?

\item
Сборные Перу и~Чили (по~$N$ игроков в~каждой) намерены сыграть серию матчей
по~борьбе, где более сильный игрок всегда побеждает более слабого.
Для каждого матча организуется $10$~пар: перуанец против чилийца, в~каждой паре
побеждает один из~соперников, счет в~матче~--- по~числу побед.
Организаторам известны сравнительные силы игроков внутри каждой из~команд,
но~не~между игроками из~разных стран.
Они собирается устраивать матчи до~тех пор, пока какой-нибудь матч
не~закончится вничью (или пока не~выяснится, что ничейный матч невозможен).
Каким наименьшим числом матчей они всегда могут обойтись?
Решите задачу для
\\
\subproblem $N = 2$;
\qquad
\subproblem $N = 4$;
\qquad
\subproblem $N = 10$.

\item
\subproblem
Имеются 300 яблок, любые два из~которых различаются по~весу не~более чем
в~два раза.
Докажите, что их можно разложить в~пакеты по~два яблока так, чтобы любые два
пакета различались по~весу не~более чем в~полтора раза.
\\
\subproblem
Имеются 300 яблок, любые два из~которых различаются по~весу не~более чем в~три
раза.
Докажите, что их можно разложить в~пакеты по~четыре яблока так, чтобы любые два
пакета различались по~весу не~более чем в~полтора раза.

\item
Есть 1000 яблок, которые надо разложить в~10 пакетов по~100 яблок в~каждом.
Оказалось, что при любой такой раскладке найдутся хотя~бы два пакета
одинакового веса.
Докажите, что
\\
\subproblem есть по~крайней мере 200 яблок одинакового веса;
\\
\subproblem есть раскладка, когда по~крайней мере 3 пакета весят одинаково.

\end{exercises}

\subsubsection*{Разбиения с небольшой разницей}

\begin{exercises}

\item\emph{Лемма о цене игры.}
На~столе лежат несколько кусков шоколада, самый большой весит $b$.
Петя начинает, и~они с~Васей по~очереди съедают по~куску, пока не~съедят всё.
Докажите, что при наилучших действиях Васи Петя сможет съесть
\\
\subproblem не~меньше Васи;
\\
\subproblem не~более чем на~$b$ больше Васи.

\item\emph{Лемма о способах выбора.}
Есть $2n$ кусков сыра.
Докажите, что можно не~менее чем $2n$ способами разложить их по~$n$ штук на~две
чаши весов так, чтобы разность весов чаш была не~больше веса самого тяжелого
куска.

\item
В~31 ящике лежит смесь апельсинов и~бананов.
Докажите, что можно так выбрать
16 ящиков, что в~них окажется по~весу не~менее половины всех апельсинов
и~не~менее половины всех бананов.

\end{exercises}

\subsubsection*{Зачетные задачи}

\begin{problems}

\item
В~31 ящике лежит смесь апельсинов и~бананов.
Докажите, что можно так выбрать
11 ящиков, что в~них окажется по~весу не~менее трети всех апельсинов
и~не~менее трети всех бананов.

\item
Есть $20$ фруктов.
Назовем натуральное число $k < 20$ \emph{хорошим,} если найдется $k$ фруктов,
чей вес равен ровно половине общего веса.
Каково наибольшее возможное количество хороших чисел?

\item
Есть 1000 яблок, которые надо разложить в~10 пакетов по~100 яблок в~каждом.
Оказалось, что при любой такой раскладке найдутся хотя~бы два пакета
одинакового веса.
Докажите, есть раскладка, когда по~крайней мере 8 пакетов весят одинаково.

\end{problems}

% www.ashap.info/Uroki/Mosbory/2018o


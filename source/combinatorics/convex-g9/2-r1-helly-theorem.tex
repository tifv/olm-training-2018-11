% $date: 2018-11-08 p1
% $date$groups:
%   g9r1: 2018-11-08 p1

% Серия 3. Теорема Хелли
\worksheet{Теорема Хелли}

% $authors:
% - Григорий Алексеевич Юргин

\begin{problems}

\item\emph{Теорема Хелли.}
\\
\subproblem
На~прямой даны несколько отрезков, любые два из~которых имеют общую точку.
Докажите, что все отрезки имеют общую точку.
\\
\subproblem
На~плоскости даны четыре выпуклые фигуры, причём любые три из~них имеют общую
точку.
Докажите, что тогда и~все они имеют общую точку.
\\
\subproblem
На~плоскости дано $n$ выпуклых фигур, причём любые три из~них имеют общую
точку.
Докажите, что все $n$~фигур имеют общую точку.

\item
Дан выпуклый многоугольник.
Известно, что для любых трёх его сторон существует такая точка~$S$ внутри
многоугольника, что проекции точки~$S$ на~прямые, содержащие эти стороны,
попадают на~сами стороны, а~не~на~их продолжения.
Докажите, что существует точка внутри многоугольника, обладающая тем~же
свойством относительно всех сторон одновременно.

\item
На~плоскости даны прямая~$\ell$ и~несколько не~обязательно выпуклых
многоугольников, каждые два из~которых имеют общую точку.
Докажите, что найдётся прямая, параллельная $\ell$ и~пересекающая все эти
многоугольники.

\item
Несколько полуплоскостей покрывают всю плоскость.
Докажите, что из~них можно выбрать не~более трёх, которые также покрывают всю
плоскость.

\item
%\subproblem
%На~плоскости даны несколько точек, любые три из~которых можно накрыть кругом
%радиуса~$1$.
%Докажите, что все точки можно накрыть кругом радиуса~$1$.
%\\
%\subproblem
\emph{Теорема Юнга.}
На~плоскости даны несколько точек, причём расстояние между любыми двумя
не~превосходит $1$.
Докажите, что все точки можно накрыть кругом радиуса $1 / \sqrt{3}$.

%\item
%Докажите, что строго внутри выпуклого $(3n + 1)$-угольника~$P$ найдется точка,
%не~содержащаяся строго внутри ни~в~одном $(n + 2)$-угольнике с~вершинами
%из~подряд идущих вершин $P$.

%\item
%На~плоскости даны $n$~точек общего положения.
%Докажите, что можно отметить на~плоскости такую точку~$O$, что по~любую сторону
%относительно любой прямой, проходящей через $O$, будет лежать не~менее
%$n/3$ точек.
%% Точно не~помню, но~возможно здесь могли возникать траблы с~версией теоремы Хелли для бесконечного семейства компактов.

%\item
%На~плоскости отмечены 2006 точек.
%Оказалось, что среди любых семи из~них есть четыре, лежащие на~одной
%окружности.
%Докажите, что найдутся хотя~бы 1003 отмеченных точки, лежащие на~одной
%окружности.

\item
\emph{Теорема Бляшке.}
Про выпуклый многоугольник~$M$ известно, что для любой прямой длина отрезка,
служащего проекцией многоугольника~$M$ на~эту прямую, не~меньше $1$.
Докажите, что $M$ заключает внутри себя круг радиуса~$1/3$.

\item
Дано несколько параллельных отрезков, причём для любых трёх из~них найдётся
прямая, их пересекающая.
Докажите, что найдётся прямая, пересекающая все эти отрезки.
% Вот это моя любимая. (Андрей).

\item
\emph{Теорема Красносельского.}
Дан не~обязательно выпуклый многоугольник~$M$.
Известно, что для любых трёх его сторон можно указать точку внутри $M$,
из~которой эти стороны видны полностью.
Докажите, что существует точка внутри $M$, из~которой видны полностью все
стороны $M$.

\end{problems}


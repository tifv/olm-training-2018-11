% $date: 2018-11-07 p3
% $date$groups:
%   g9r1: 2018-11-07 p3

% Серия 2. Выпуклость
\worksheet{Выпуклость}

% $authors:
% - Григорий Алексеевич Юргин

\begin{problems}

\item
На~круглой сковороде площади~$1$ испекли выпуклый блин (многоугольник) площади
больше $1/2$.
Докажите, что центр сковороды находится под блином.

\item
На~плоскости даны $n \geq 4$ точек.
Известно, что любые $4$~точки являются вершинами некоторого выпуклого
четырёхугольника.
Докажите, что эти $n$~точек являются вершинами некоторого выпуклого
$n$-угольника.

\item
На~плоскости даны несколько правильных $n$-угольников.
Докажите, что выпуклая оболочка объединения всех этих $n$-угольников имеет
не~менее $n$~вершин.

%\item
%Выпуклый фанерный многоугольник~$P$ лежит на~деревянном столе.
%В~стол можно вбивать гвозди, которые не~должны проходить через $P$, но~могут
%касаться его границы.
%Фиксирующим называется набор гвоздей, не~позволяющий двигать $P$ по~столу.
%Найдите минимальное количество гвоздей, позволяющее зафиксировать любой
%выпуклый многоугольник.

\item
На~плоскости даны три попарно непересекающихся выпуклых многоугольника.
Докажите, что следующие два утверждения равносильны.
\begin{itemize}\parskip=0.5ex
\item
Не~существует прямой, пересекающей все три многоугольника.
\item
Любой из~многоугольников можно отделить прямой от~двух других.
\end{itemize}

\item
На~плоскости нарисовано несколько прямых общего положения, по~каждой из~которых
со~скоростью~$1$ ползёт жук.
Докажите, что в~некоторый момент времени жуки окажутся в~вершинах некоторого
выпуклого многоугольника.
% Эта мне очень нравится, но возможны очень неприятные решения от~детей.
% (Андрей)

\item
Докажите, что выпуклый многоугольник площади~$1$ можно поместить в~некоторый
прямоугольник площади~$2$.
% Вот эта мне нравится. Классика по теме. (Андрей)

\item
Определите минимальное натуральное число~$k$ такое, что любой выпуклый
$100$-угольник можно получить пересечением $k$~треугольников.

%\item
%Внутрь выпуклого центрально симметричного многоугольника~$P$ помещён
%треугольник~$T$, внутри треугольника отмечена точка~$S$.
%Докажите, что хотя~бы одна из~вершин треугольника~$T$ после симметрии
%относительно $S$ останется внутри $P$.
%% Задача немного посложнее, но не гроб. (Андрей)

\end{problems}

